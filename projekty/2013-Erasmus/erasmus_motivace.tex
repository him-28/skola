\documentclass[12pt,a4paper,oneside,final]{article}
\usepackage[utf8]{inputenc} 
\usepackage[czech]{babel}
\usepackage[margin=2cm, includefoot]{geometry}
\usepackage[plainpages=false,pdfpagelabels,unicode]{hyperref}

\title{Motivační dopis pro pracovní stáž v programu Erasmus}
\date{\vspace{-5ex}}
%\date{}  % Toggle commenting to test
\pagenumbering{gobble}

\begin{document}

\maketitle


\begin{tabular}{l l}
Jméno: & Pavel Ondračka\\
Ročník a obor: & 4. semestr, 2. ročník, Magisterský studijní program Fyzika plazmatu\\
Termín stáže: & 1.9.2013 - 30.8.2014\\
\end{tabular}

\vspace{1cm}
\noindent{Vážená komise,}
\\
jmenuji se Pavel Ondračka a v současné době jsme studentem druhého ročníku navazujícího magisterského programu Fyzika plazmatu. Ve svém vzdělávání chci dále pokračovat v doktorském studijním programu Pokročilé materiály a nanovědy (studijní obor Pokročilé mikrotechnologie a nanovědy), který byl nedávno akreditován v rámci spolupráce mezi MU, VUT a CEITEC. Ve své bakalářské a nyní i diplomové práci na Masarykově Universitě se zabývám optickou charakterizací tenkých vrstev a studiem disperzních modelů pevných látek v rámci pracovní skupiny doc. Mgr. Lenky Zajíčkové Ph.D. (Ústav fyzikální elektroniky PřF a CEITEC Plasma Technologies) a v tomto odvětví bych se chtěl specializovat i nadále. 

Aplikované disperzní modely jsou založeny na parametrizaci hustoty stavů a přinášejí tedy informace o elektronové struktuře materiálů. Logickým krokem by tedy bylo porovnání experimentálně dosažených výsledků (dielektrické funkce) s teoretickými ab initio výpočty. Z tohoto důvodu bych velmi rád pracoval na Montanuniversitat Loeben (MUL) pod vedením Dr. Davida Holce na výpočtech struktury a elektronických vlastností vybraných oxidových materiálů pomocí Density Functional Theory (DFT). Tyto výpočty budou pak v rámci rozbíhající se spolupráce mezi MU a MUL porovnány s experimentálně zjištěnou dielektrickou odezvou materiálů a budou východiskem pro tvorbu nových disperzních modelů.

Hlavní přínos pracovní stáže pro mě bude rozšíření znalostí o struktuře materiálů a jejich elektronických vlastnostech, což jsou klíčové informace pro pochopení experimentálních dat získaných při jejich charakterizaci. Další velký přínos bude seznámení se s mezinárodním prostředím renomované rakouské university a rozšíření mých jazykových schopností, co se anglického a německého jazyka týče. Již během stáže, případně po mém návratu ve zbylé části doktorského studia také využiji své nově získané poznatky při optické charakterizaci tenkých vrstev připravených na MU pracovní skupinou Lenky Zajíčkové. Nezanedbatelný je také přínos pracovní stáže pro moje budoucí uplatnění na trhu práce po ukončení studia.
\newline
\begin{flushright}
\begin{tabular}{ll}
S pozdravem \\
Pavel Ondračka
\end{tabular}
\end{flushright}
\vspace{0,3cm}
\begin{tabular}{ll}
V Brně dne: \today \hspace{5cm} 
\end{tabular}




\end{document}
