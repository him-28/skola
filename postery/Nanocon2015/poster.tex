\documentclass[blues]{poster}
\usepackage[A1,portrait]{vmargin}
\usepackage[english]{babel}
\usepackage[mathletters]{ucs}
\usepackage[utf8x]{inputenc}
\usepackage{wrapfig}
%\usepackage{tikz}

\let\eps=\varepsilon
%\let\phi=\varphi
%\let\rho=\varrho
%\def\theta{{\vartheta₀}}
%\def\xstrut{\vrule width0pt height2.5ex depth1.0ex\relax}

\makeatletter
\@definecounter{romlist}
\def\romlist{\edef\@romlistctr{romlist}
    \par\unskip
    \begingroup
    \parskip=0pt
    \list{\csname label\@romlistctr\endcsname}
         {\usecounter{\@romlistctr}%
          \def\makelabel##1{\hss\llap{##1}}}%
          \setlength\itemsep{0pt}%
          \setlength\parsep{0pt}%
       }
\def\endromlist{\endlist\endgroup}
\def\labelromlist{\textbullet}
\def\theromlist{\arabic{romlist}}
\def\@authorboxwidth{50cm}
\makeatother

\setmargnohfrb{1.8cm}{1.8cm}{1.8cm}{1.8cm}%
\newcommand{\SiTiO}{Si$_x$Ti$_{1-x}$O$_2$}

\begin{document}
\title{Composition induced changes in optical response of Ti$_{1-x}$Si$_x$O$_2$ from first priciples calculations}
\affiliation
  {\hsize=18cm
   \hbox{\includegraphics[height=4.2cm]{MU-blue.pdf} \quad
	\includegraphics[height=3.8cm]{OPVK_hor_zakladni_logolink_RGB_eng.jpg}}
   \hbox{\includegraphics[height=4.2cm]{MUL.jpeg}\quad
		 \includegraphics[height=4.0cm]{University_of_Nantes.pdf}
         \includegraphics[height=4.2cm]{CEITEC.pdf}}
	}
  {\authorbox
   {Pavel Ondračka $^{a,b}$, David Holec $^c$, Daniel Franta $^{a}$,\\
	Eva Kedroňová $^{a,b}$, Stéphane Elisabeth $^d$,\\
	Marek Eliáš $^{a,b}$, Antoine Goullet $^d$, Lenka Zajíčková $^{a,b}$}
   {$^a$ Faculty of Science, Masaryk University, Kotlářská 2, 611 37 Brno, Czech Republic\\
	$^b$ CEITEC - Central European Institute of Technology, Masaryk University, Kotlářská 2, 611 37 Brno, Czech Republic\\
	$^c$ Department of Physical Metallurgy and Materials Testing, Montanuniversität Leoben,\\
		Franz-Josef-Straße 18, Leoben A-8700, Austria\\
	$^d$ Institut des Matériaux Jean Rouxel (IMN), Université de Nantes,\\
		UMR CNRS 6502, 2 rue de la Houssinière, BP 32229, 44322 Nantes Cedex 3, France
	}
}

\vskip-1ex
\hrule height0pt

\begin{multicols}{2}

\subsection{Introduction and methodology }

\begin{wrapfigure}{r}{0.30\linewidth}
  \vspace{-20pt}
  \begin{center}
    \includegraphics[height=3cm]{WIEN2k-logo.jpg}
    \includegraphics[height=3cm]{VASP-logo.jpg}
  \end{center}
  \vspace{-20pt}
\end{wrapfigure}

Mixed Ti$_{1-x}$Si$_x$O$_2$ materials offer interesting optical properties due to the high difference between refractive indices (TiO$_2$: $\sim$2.5, SiO$_2$: $\sim$1.5) and band gaps (TiO$_2$: $\sim$3.2\,eV, SiO$_2$: $\sim$8.5\,eV). Mixing of SiO$_2$ and TiO$_2$ is a perspective way how to alter the dielectric constant and refractive index, and simultaneously improve insulating properties. This opens new possibilities in designing optical devices such as multilayer antireflective coatings, interference filters, or waveguides. Alloying Si into TiO$_2$ can also be used to overcome some of TiO$_2$ films shortcomings, such as the columnar morphology, which leads to increased optical losses and degradation of its insulating character.

In the present work, the variation of Ti$_{1-x}$Si$_x$O$_2$ optical constants with the Si concentration are examined by employing \emph{Density Functional Theory}. 
The \emph{Special Quasi-random Structure} method is used to generate structural models of Si$_x$Ti$_{(1-x)}$O$_2$ disordered states for $x$=0.0625, 0.125, 0.1875, 0.25, 0.5, 0.75 and 1 for \emph{anatase} and \emph{rutile} phases. These initial supercells are structurally optimized using the Vienna \textit{Ab initio} Simulation Package. 
Optical constants of the resulting structures are calculated by the linearized augmented plane wave method as implemented in the Wien2k full potential all electron code together with the recently developed \emph{modified Becke-Johnson exchange-correlation potential} (mBJ).

Calculations are compared with experimental data obtained by \emph{ellipsometry and spectrophotometry} performed on Si$_x$Ti$_{(1-x)}$O$_2$ samples deposited by plasma-enhanced chemical vapor deposition (PECVD) and atomic player deposition (ALD). Stochiometry of deposited films was determined by XPS.

\subsection{Band gap evolution}

\begin{minipage}{0.4\linewidth}

\begin{itemize}
\item{Much better agreement between experimental and calculated band gap with mBJ compared to conventional LDA or GGA}
\item{Bang gap decreases slightly between $x = 0$ and $x = 0.5$, then it increases sharply}
\item{Calculated band gap evolution trend matches our experimental data}
\end{itemize}

\end{minipage}
\begin{minipage}{0.6\linewidth}   

\includegraphics[width=\linewidth]{figures/gap.pdf}
\end{minipage}


\vspace{-0.5cm}
\subsection{Band structure evolution}
\includegraphics[width=\hsize]{figures/SiTiO2-dos.pdf}

\subsection{Evolution of optical constants}

\begin{minipage}{0.4\linewidth}

\begin{itemize}
\item{Calculated values of refractive index at 400\,nm wavelength are overestimated with respect to the experimental data}
\item{This is due to the overestimated second absorption peak around 7\,eV}
\item{Predicted trend agrees well with the experiment}
\end{itemize}

\end{minipage}
\begin{minipage}{0.6\linewidth}   
\includegraphics[width=\hsize]{figures/n-anatase.pdf}
\end{minipage}

\includegraphics[width=\hsize]{figures/SiTiO2-eps.pdf}

\includegraphics[width=\hsize]{figures/epsi.pdf}

\includegraphics[width=\hsize]{figures/compare.pdf}

\begin{itemize}
\item{Very good match between experimental and calculated anatase $\eps_\mathrm{i}$ for $x = 0$ and $E < 6$\,eV}
\item{Calculated $\eps_\mathrm{i}$ is too high for energies above 6\,eV and $x > 0$ which results in mismatch of experimental and calculated refractive index}
\end{itemize}

\subsection{Dielectric function near absorption onset}

\begin{minipage}{0.5\linewidth}   

\begin{itemize}
\item{Downshift of the absorption edge with increasing Si content}
\item{Very weak absorption near the absorption edge could explain the difference between calculated and experimental band gap}
\item{Silicon-induced oxygen defect states appear at the top and bottom of the valence band}
\end{itemize}

\end{minipage}
\begin{minipage}{0.5\linewidth}
\includegraphics[width=\linewidth]{figures/near-gap-eps.pdf}
\end{minipage}

\vspace{-0.5cm}
\subsection{Conclusions}
\begin{itemize}
\item{mBJ potential results in a good agreement between the calculated and experimental band gap.}
\item{A good match between the calculated and experimental $\eps_\mathrm{i}$ especially for $x = 0$ and $E < 6$\,eV.}
\item{Interesting behavior near the absorption edge due to the silicon induced oxygen states.}
\end{itemize}

\vspace{-0.5cm}
\subsection{Acknowledgment} 
This work was supported by the project ``CEITEC - Central European Institute of Technology'' (CZ.1.05/\-1.1.00/02.0068) from European Regional Development Fund, by the MOBILITY project 7AMB15AT017 funded by Czech Ministry of Education, Youth and Sports (CMEYS), by the project CZ09/2015 funded by the Austrian Agency for International Cooperation in Education and Research (OeAD-GmbH), and by the IT4Innovations Centre of Excellence project (CZ.1.05/1.1.00/02.0070) funded by the European Regional Development Fund and the national budget of the Czech Republic via the Research and Development for Innovations Operational Programme, as well as CMEYS via the project Large Research, Development and Innovations Infrastructures (LM2011033).

\end{multicols}
\end{document}
