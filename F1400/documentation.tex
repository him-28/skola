\documentclass[12pt]{article}
\usepackage[czech]{babel}
\usepackage[utf8]{inputenc}
\usepackage[plainpages=false,pdfpagelabels,unicode]{hyperref}
\usepackage[pdftex]{graphicx}
\usepackage[margin=3cm, includefoot]{geometry}

\begin{document}

\title{Dokumentace k~zápočtovému programu \\[2cm]
	Úloha 2.\\ Zpracování obrazu: analýza objektů}
\author{Pavel Ondračka}

\maketitle
\clearpage

\tableofcontents
\clearpage



\section{Zadání}

\subsection{Původní zadání}
Vytvořte soubor, který bude obsahovat matici nul a jedniček. Shluky jedniček budou reprezentovat nějaké objekty, například kolečka či obdélníčky. Tento soubor vygenerujte Octavem, ručně nebo jinak. Napište program v~Céčku, který tento obrázek přečte, separuje ho na~jednotlivé objekty, u~každého z~nich spočítá obsah, obvod a tyto vypíše do~tabulky. Vyzkoušejte pro~různé vstupní soubory (obrázky).

\subsection{Změny v~zadání}
Na příkladu mi přišly nejzajímavější rozpoznávací algoritmy, výsledný program proto umí rozpoznávat tři typy obrazců: čtverce, obdélníky (jejichž strany jsou vodorovné a svislé) a čtverce otočené o~$\pi/4$ a to jak samostatné, tak v~ideálním případě i pokud se~částečně překrývají. Program na~výstup vypíše pozice a velikosti těchto objektů. Dále byl pokaždé vypočítán obsah nalezeného obrazce.

\section{Generace vstupního souboru}
\subsection{Program generate.c}
Program generate vypíše na~výstup pole znaků o~velikosti $n$ řádků a sloupců ($n$ je definováno konstantou PICTURE\_SIZE v~souboru util.c), tento výstup obsahuje libovolný počet čtverců, otočených čtverců a obdélníků (definováno v~záhlaví souboru generate.c). Tyto se~vypíšou pomocí funkcí draw\_rectangle() a draw\_rotated\_square() (budou popsány níže). Dále lze měnit znaky kterým jsou vypsány objekty a znaky pozadí a to pomocí konstant PSYMBOL a PBACKGROUND. Program každý obrazec vykreslí na~náhodné umístění tak, aby obrazec nepřečníval z~obrázku a mezi~krajem a obrazcem byl alespoň jeden prázdný pixel.

\section{Průběh rozpoznávání}
\subsection{Samostatné objekty}
Program read.c nejprve načte vstupní soubor do~pole cpicture jako sekvenci nul a jedniček, následně jej pro~kontrolu vypíše na~obrazovku. Poté začne postupně po~řádcích testovat cpicture a v~momentě kdy najde první neprázdný znak, zavolá po~řadě rozpoznávací funkce check\_rotated\_square(), check\_square() a check\_rectangle(). Pokud tyto funkce naleznou příslušný obrazec (pouze samostatný), vypíšou toto na~obrazovku a obrazec z~cpicture vymažou. Mažeme proto abychom již dále netestovali body které máme už přiřazené. Po~rozpoznání všech samostatných obrazců přichází na~řadu rozpoznávání sloučených objektů.

\subsection{Překrývající se~objekty}
Máme 3 stejná pole ve kterých jsou zbývající obrazce: cpicture, cpicture a cpicture2. Picture je nyní konstantní a slouží nám jako reference, v~cpicture a pictuje2 budeme testovat obrazce. pixelcount je aktuální počet pixelů, které ještě nemáme přiřazené žádnému obrazci. Najdeme tedy v~cpicture první nenulový pixel. do~cpicture2 uložíme cpicture aby byly oba dva stejné. Nyní do~cpicture postupně generujeme obrazce s~postupně rostoucími rozměry, tyto testujeme proti cpicture. Pokud přečnívají, zkoušíme dál, pokud ne, sečteme aktuální počet nepřiřazených pixelů v~cpicture a pokud je menší než pixelcount, obrazec uložíme pomocí funkce save\_object() do~pole objects. Funkce save\_objects() ukládá jen jeden nejlepší obrazec pro~každý testovaný pixel, t.j. po~otestování všech obrazců zůstane ten, který do~daného bodu sedí nejlépe a tento obrazec pak z~cpicture vymažeme.

\section{Popis jednotlivých funkcí}
\subsection{check\_rotated\_square()}
Na vstupu je pole znaků, a bod [i, j] \footnote{vzhledem k~tomu že čteme vstupní soubor zleva-doprava a shora-dolů, je $i$ číslo řádku s~hora a $j$ je číslo sloupce z~leva (obě začínají nulou)} kde začínáme testovat. Máme dva cykly, s~proměnnými $k$ a $l$. $k$ udává který řádek testujeme, $l$ je pro~sloupce. $k$ jde od nuly do~maximální velikosti obrázku, případně pokud jsme už určili $r$ tak do~$r$. $l$ jde buď do~maximální velikosti obrázku nebo $2r - k$. pro~$r$ platí, že $2r + 1$ je velikost úhlopříčky. pro~každý řádek $i+k$ tedy musí být zaplněny pole $j-k, j-k+1, ...., j+k-1, j+k$ (toto platí pouze dokud ještě neznáme úhlopříčku čtverce a on se~postupně zvětšuje). Také musí platit, že $j-k-1$ a $j+k+1$ musí být nula (obrazec je samostatný). Pokud nalezneme řádek kde pole $j-k$ a $j+k$ jsou prázdné, pole $j-k+1$ a $j+k-1$ také, ale všechna pole mezi~jsou plná, znamená to, že jsme našli bod kde se~čtverec začíná zmenšovat, uložíme $r$ a dokreslíme obrazec. Nakonec nalezený obrazec z~cpicture vymažeme, údaje o~něm vypíšeme na~obrazovku a vrátíme 1. Pokud se~nic nenašlo, funkce vrací 0.

\subsection{check\_\{rectangle, square\}() }
Funkce jsou podobné check\_rotated\_square(), hlavní cyklus vypadá podobně, opět dokud neznáme velikosti stran tak testujeme v~rozmezí velikosti obrázku, v~případě úspěchu vypíšeme na~obrazovku údaje o~objektu a vrátíme 0 pro~neúspěch nebo jedničku pro~úspěch. Opět jsou také testovány okolní pixely, jestli je obrazec osamocený.

\subsection{draw\_\{rectangle,rotated\_square\}()}
Tyto funkce vykreslí zadaný objekt do~obrázku. na~vstupu je pole s~obrázkem, poloha objektu, jeho geometrické vlastnosti a číslo, kterým chceme objekt v~obrázku reprezentovat. Kromě kreslení objektů je funkce draw\_rectangle také několikrát použita k~inicializace polí. 

\subsection{Ostatní funkce}
Ostatní funkce použité v~programu nejsou z~hlediska samotného výpočtu důležité a slouží hlavně pro~zmenšení počtu řádků. Patří mezi~ně funkce random\_number(a,b), která generuje náhodné celé číslo v~mezích [$a,b$], funkce compare porovnává dva obrázky na~rozdíly, save\_object() ukládá nalezené překrývající se~objekty, count\_pixels($a$) sečte hodnoty všech buněk v~poli $a$, sync\_pictures($a,b$) zkopíruje obsah pole $b$ do~pole $a$. Draw\_last\_object($objects$) vykreslí poslední obrazec v~poli $objects$. Funkce display\_objects() vypíše na~výstup parametry všech nalezených překrývajících se objektů.


\section{Příklady použití}
\subsection{Kompilace}

\begin{verbatim}
$ gcc generate.c -o generate
$ gcc read.c -o read
\end{verbatim}

\subsection{Návod k~použití}
Vygenerujeme náhodný obrázek a uložíme ho do~souboru test.txt
\begin{verbatim}
$ ./generate > test.txt
\end{verbatim}

Následně obrázek otestujeme programem read.

\begin{verbatim}
$ ./read < test.txt
\end{verbatim}

Zkontrolujeme výstup.

\subsection{Příklady}
Všechny tyto příklady byly vygenerovány programem generate. Některé byly pro~ušetření místa na~krajích ručně ořezány. 

\subsubsection{Příklad 1 -- jednoduchý obdélník}

\paragraph{Vstup}
\begin{verbatim}
----------
-xxxxxxxx-
-xxxxxxxx-
-xxxxxxxx-
-xxxxxxxx-
-xxxxxxxx-
----------
\end{verbatim}
\paragraph{Výstup}
\begin{verbatim}
Nalezen obdélník: řádek:1, sloupec:1, šířka:8, výška:5, obsah:40 

VÝSLEDKY
Samostatné obrazce:
Nalezeno otočených čtverců:0, čtverců:0, obdélníků:1
Překrývající se obrazce:
Nalezeno otočených čtverců:0, čtverců:0, obdélníků:0
\end{verbatim}

\subsubsection{Příklad 2 -- samostatný čtverec}

\paragraph{Vstup}
\begin{verbatim}
-------
-xxxxx-
-xxxxx-
-xxxxx-
-xxxxx-
-xxxxx-
-------
\end{verbatim}
\paragraph{Výstup}
\begin{verbatim}
Nalezen čtverec: řádek:1 sloupec:1, velikost:5, obsah:25 

VÝSLEDKY
Samostatné obrazce:
Nalezeno otočených čtverců:0, čtverců:1, obdélníků:0
Překrývající se obrazce:
Nalezeno otočených čtverců:0, čtverců:0, obdélníků:0
\end{verbatim}

\subsubsection{Příklad 3 -- samostatný otočený čtverec}
\paragraph{Vstup}
\begin{verbatim}
-------
---x---
--xxx--
-xxxxx-
--xxx--
---x---
-------
\end{verbatim}
\paragraph{Výstup}
\begin{verbatim}
Nalezen otočený čtverec: řádek:1, sloupec:3, úhlopříčka:5, obsah:13

VÝSLEDKY
Samostatné obrazce:
Nalezeno otočených čtverců:1, čtverců:0, obdélníků:0
Překrývající se obrazce:
Nalezeno otočených čtverců:0, čtverců:0, obdélníků:0
\end{verbatim}

\subsubsection{Příklad 4 -- překrývající~se otočené čtverce}
\paragraph{Vstup}
\begin{verbatim}
-------------
--------x----
-------xxx---
----x-xxxxx--
---xxxxxxxxx-
--xxxxxxxxx--
-xxxxxxxxx---
--xxxxx-x----
---xxx-------
----x--------
-------------
\end{verbatim}
\paragraph{Výstup}
\begin{verbatim}
Nalezen otočený čtverec: řádek:1, sloupec:8, úhlopříčka:7, objem:25
Nalezen otočený čtverec: řádek:3, sloupec:4, úhlopříčka:7, objem:25

VÝSLEDKY
Samostatné obrazce:
Nalezeno otočených čtverců:0, čtverců:0, obdélníků:0
Překrývající se obrazce:
Nalezeno otočených čtverců:2, čtverců:0, obdélníků:0
\end{verbatim}

\subsubsection{Příklad 5 -- překrývající se~obdélníky}
\paragraph{Vstup}
\begin{verbatim}
----------
--xxxxxx--
-xxxxxxxx-
-xxxxxxxx-
-xxxxxxxx-
--xxxxxx--
----------
\end{verbatim}
\paragraph{Výstup}
\begin{verbatim}
Nalezen obdélník: řádek:1, sloupec:2, výška:5, šířka:6, objem:30 
Nalezen obdélník: řádek:2, sloupec:1, výška:3, šířka:8, objem:24 

VÝSLEDKY
Samostatné obrazce:
Nalezeno otočených čtverců:0, čtverců:0, obdélníků:0
Překrývající se obrazce:
Nalezeno otočených čtverců:0, čtverců:0, obdélníků:2
\end{verbatim}

\subsubsection{Příklad 6 -- překrývající se~obdélníky 2}
\paragraph{Vstup}
\begin{verbatim}
----------
-xxxxxxxx-
-xxxxxxxx-
-xxxxxxxx-
-xxxxxxxx-
--xxxxxx--
----------
\end{verbatim}
\paragraph{Výstup}
\begin{verbatim}
Nalezen obdélník: řádek:1, sloupec:1, šířka:8, výška:4, obsah:32 

VÝSLEDKY
Samostatné obrazce:
Nalezeno otočených čtverců:0, čtverců:0, obdélníků:1
Překrývající se obrazce:
Nalezeno otočených čtverců:0, čtverců:0, obdélníků:0
\end{verbatim}

\subsubsection{Příklad 7 -- překrývající se~čtverce}
\paragraph{Vstup}
\begin{verbatim}
--------
-xxxx---
-xxxxxx-
-xxxxxx-
-xxxxxx-
--xxxxx-
--xxxxx-
--------
\end{verbatim}
\paragraph{Výstup}
\begin{verbatim}
Nalezen čtverec: řádek:1, sloupec:1, velikost:4, objem:16 
Nalezen obdélník: řádek:2, sloupec:5, výška:5, šířka:2, objem:10 
Nalezen obdélník: řádek:5, sloupec:2, výška:2, šířka:3, objem:6 

VÝSLEDKY
Samostatné obrazce:
Nalezeno otočených čtverců:0, čtverců:0, obdélníků:0
Překrývající se obrazce:
Nalezeno otočených čtverců:0, čtverců:1, obdélníků:2
\end{verbatim}

\section{Shrnutí výsledků a závěr}
Jak je vidět z~testů, program je 100\,\% úspěšný při~rozpoznávání samostatných obrazců. Úspěšnost při~rozpoznávání překrývajících se~objektů je ale o~něco horší. To je způsobeno použitým nedokonalým algoritmem. Hlavní problém je ten, že program nalezené objekty vymaže z~testovacího obrázku (cpicture). Výhodou tohoto postupu je nesrovnatelně větší rychlost\footnote{pro malé obrázky z~testů je to zanedbatelné, ale už při~obrázku $100 \times 100$ se~jedná řádově o~vteřiny}, nevýhodou je to, že se~obrázky generují z~levého horního rohu (respektive z~vrchu u~otočených čtverců). Tedy pokud vymažeme tento startovní bod nebude obrazec dobře rozpoznán. Možné řešení tohoto problému by bylo generovat testovací obrazce do~všech možných směrů. Další nevýhodou algoritmu je to, že někdy může dávat odlišné výsledky než jaké čekáme. Dobrou ukázkou je příklad 7, kde program najde čtverec a dva obdélníky tam kde většina uživatelů uvidí dva čtverce\footnote{respektive viděla by kdyby obrázky byly v~poměru, t.j. šířka a výška znaku by byla stejná}. Další možnost řešení tohoto problému by bylo ukládat všechny možné obrazce a nástedně je testovat mezi~sebou na~podmonožiny. Toto by ale bylo ještě o~mnoho řádů výpočetně náročnější.

Co se~týče možných optimalizací. Kromě možného výše zmíněného zlepšení je v~programu jistě mnoho dalších možností na~zlepšení. Pravděpodobně by se~dala optimalizovat práce s~poli, také by bylo možné požívat menší typy pro~proměnné, a určitě by šlo více sdílet kód mezi~jednotlivými funkcemi (příkladem duplicitního kódu jsou například funkce check\_square() a check\_rectangle() ). Nicméně si myslím, že program jako takový splňuje požadavky a dobře ilustruje možnosti strojového rozpoznávání obrazců. 



\end{document}
