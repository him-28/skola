\documentclass[10pt,a4paper,twocolumn]{article}
\usepackage[english]{babel}
\usepackage[utf8]{inputenc}
\usepackage[plainpages=false,pdfpagelabels,unicode]{hyperref}
\usepackage[pdftex]{graphicx}
\usepackage[margin=1.5cm, includefoot]{geometry}
\usepackage{authblk}
\usepackage{xspace}
\usepackage[numbers,sort&compress]{natbib}

\author[a,b,c]{Pavel Ondračka}
\author[c]{David Holec}
\author[a,b]{Lenka Zajíčková}
\affil[a]{Faculty of Science, Masaryk University, Kotlářská 2, 611 37 Brno, Czech Republic}
\affil[b]{CEITEC - Central European Institute of Technology, Masaryk University, Kotlářská 2, 611 37 Brno, Czech Republic}
\affil[c]{Department of Physical Metallurgy and Materials Testing, Montanuniversität Leoben, Franz-Josef-Straße 18, Leoben A-8700, Austria}

\title{Optical properties of monoclinic, cubic, tetragonal and amorphous HfO2 with TB-mBJ}
\date{}

\begin{document}

\twocolumn[
  \begin{@twocolumnfalse}
    \maketitle
    \begin{abstract}    
    
    \end{abstract}
  \end{@twocolumnfalse}
]

\section{Introduction}
Something about HfO2 polymorphs, transform temperatures etc, wide band gap etc...

HfO$_2$ is attracting a lot of attention as a high-$k$ material for electronic applications as well as optical applications such as antirelective coatings~\cite{Fadel1998, Khoshman2008} or ....). 


Well calculated band gap and band staructure is critical for correct prediction of optical properties.

On the other hand, hybrid functionals or other more sophisticated methods as GW, give band gaps in much better agreament with experiment, however are much more computionally expansive. TM-mBJ 

Monoclinic HfO2 was used during callibration of the original TB-mBJ parameters and a good aggreement was obtained (....)~\cite{Tran2009}, it is however unclear, if this extends also to other forms of HfO$_2$.
This problem can be seen in case of TiO$_2$, where we have a very good aggreament in calculated band gap for anatase, however the difference for calculated and experimental values for rutile is almost twice as big \cite{Sai2012}.  


%The latter is known to be underestimated by conventional approximations for the exchange-correlation (xc) potential, local density approximation (LDA) and generalised gradient approximation (GGA). 
%Recently, \citet{Tran2009} have proposed a new semi-local xc potential (modified Becke-Johnson, mBJ) which yields highly accurate energy band gaps in most semiconductors and insulators~\cite{Singh2010}.
%The clear improvement towards experimental measurements using the mBJ potential instead of the conventional GGA was demonstrated by~\citet{Sai2012}.
%The huge impact of the improved description of the electronic structure on the predicted optical properties, such as refractive index, $n$, extinction coefficient, $\kappa$, or dielectric function $\epsilon$, when employing the mBJ potential was shown by \citet{Sai2012} for the case of rutile and anatase TiO$_2$.

In this work, we test the applicability of TB-mBJ potential for monoclinic, tertagonal, cubic and amorphous HfO2 with focus on band gap and optical properties.

\section{Methodology}

As TB-mBJ in not a FIXME, it is not suitable for optimisation of structural properties, hence experimental values were used for monoclinic, tetragonal and cubic cells. Amorphous unit cell was prepared by a simulated annealing  This was done using the the Vienna Ab initio Simulation Package~\cite{Kresse1996}, with projector augmented pseudopotentials \cite{Kresse1999} and using the generalised gradient approximation  parametrized by Perdew, Burke and Ernzerhof (GGA-PBE) \cite{Perdew1996}.

%Electronic structure and optical constants were then calculated using the linearized augmented plane wave method as implemented in the Wien2k full potential all electron code~\cite{Blaha2001} together with the modified Becke-Johnson (mBJ) potential \cite{Tran2009} for exchange and GGA-PBE for correlation effects allowing precise prediction of electronic structure and band gap.

For optical properties we used the optic code~\cite{AmbroschDraxl2006}, a part of the Wien2k package.
 
%Lorenz broadening of 0.1\,eV was used for dielectric function in order to better match room temperature experimental optical measurements.  

\section{Results a discussion}

\section{Conclusions}

\bibliographystyle{unsrtnat}
\bibliography{bib-db}

\end{document}
