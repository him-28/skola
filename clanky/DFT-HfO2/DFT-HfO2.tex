\documentclass[10pt,a4paper,twocolumn]{article}
\usepackage[english]{babel}
\usepackage[utf8]{inputenc}
\usepackage[plainpages=false,pdfpagelabels,unicode]{hyperref}
\usepackage[pdftex]{graphicx}
\usepackage[margin=1.5cm, includefoot]{geometry}
\usepackage{authblk}
\usepackage{xspace}
\usepackage[numbers,sort&compress]{natbib}

\author[a,b,c]{Pavel Ondračka}
\author[c]{David Holec}
\author[a,b]{Lenka Zajíčková}
\affil[a]{Faculty of Science, Masaryk University, Kotlářská 2, 611 37 Brno, Czech Republic}
\affil[b]{CEITEC - Central European Institute of Technology, Masaryk University, Kotlářská 2, 611 37 Brno, Czech Republic}
\affil[c]{Department of Physical Metallurgy and Materials Testing, Montanuniversität Leoben, Franz-Josef-Straße 18, Leoben A-8700, Austria}

\title{Optical properties of monoclinic, cubic, tetragonal and amorphous HfO2 with TB-mBJ}
\date{}

\begin{document}

\twocolumn[
  \begin{@twocolumnfalse}
    \maketitle
    \begin{abstract}    
    
    \end{abstract}
  \end{@twocolumnfalse}
]

\section{Introduction}
Something about HfO$_2$ polymorphs, transform temperatures etc, wide band gap etc...

HfO$_2$ is attracting a lot of attention as a high-$k$ material for electronic applications as well as optical applications such as antireflective coatings~\cite{Fadel1998, Khoshman2008}, heat-mirrors~\cite{Al-Kuhaili2004}, or laser mirrors~\cite{Meng2012}). 

Well calculated band gap and band structure is critical for correct prediction of optical properties. 

Majority of the HfO$_2$ ab initio studies employ conventional approximations for the exchange-correlation (xc) potential, local density approximation (LDA) and generalised gradient approximation (GGA).

On the other hand, hybrid functionals or other more sophisticated methods as GW, give band gaps in much better agreament with experiment, however are much more computationally expansive.
A recently developed semi-local xc potential (modified Becke-Johnson, TB-mBJ) is an alternative that can provide highly accurate band gaps at the computational cost of LDA or GGA~\cite{Tran2009}.

Band gap of monoclinic HfO$_2$ was already found to be 5.83\,eV using the original TB-mBJ parameters~\cite{Koller2012}.
This is in a good agreement with the experimental value of .... \cite{}, it is however unclear, if this extends also to other forms of HfO$_2$.
This problem can be seen in case of TiO$_2$, where we have a very good agreement in calculated band gap for anatase, however the difference for calculated and experimental values for rutile is almost twice as big \cite{Sai2012}.  

%The huge impact of the improved description of the electronic structure on the predicted optical properties, such as refractive index, $n$, extinction coefficient, $\kappa$, or dielectric function $\epsilon$, when employing the mBJ potential was shown by \citet{Sai2012} for the case of rutile and anatase TiO$_2$.

In this work, we test the applicability of TB-mBJ potential for monoclinic, (tetragonal), cubic and amorphous HfO$_2$ with focus on band gap and optical properties.

\section{Methodology}

There is no associated exchange functional for TB-mBJ, hence its not suitable for structural optimization.
We used experimental values for structural parameters of HfO$_2$.
For m-HfO$_2$ values of \citet{Adam1959} ($a = 5.1156$\,\r{A}, $b = 5.1722$\,\r{A}, $c = 5.2948$\,\r{A}, $\beta = 99.11^\circ$).
%FIXME: the beta from Adam is different, also according to OCD the constants are from Ruh, but in overview Ruh has different...
c-HfO$_2$ by \citet{Dole1978277} ($a = 5.115$\,\r{A}).
Amorphous unit cell was prepared by a simulated annealing FIXME: .
This was done using the the Vienna Ab initio Simulation Package~\cite{Kresse1996}, with projector augmented pseudopotentials \cite{Kresse1999} and using the generalised gradient approximation parametrized by Perdew, Burke and Ernzerhof (GGA-PBE) \cite{Perdew1996}. 

Optical properties and band gaps were calculated using the Wien2k full potential all electron code~\cite{Blaha2001} with TB-mBJ

%Electronic structure and optical constants were then calculated using the linearized augmented plane wave method as implemented in the Wien2k full potential all electron code~\cite{Blaha2001} together with the modified Becke-Johnson (mBJ) potential \cite{Tran2009} for exchange and GGA-PBE for correlation effects allowing precise prediction of electronic structure and band gap.

For optical properties we used the optic code~\cite{AmbroschDraxl2006}, a part of the Wien2k package.
 
%Lorenz broadening of 0.1\,eV was used for dielectric function in order to better match room temperature experimental optical measurements.  

\section{Results a discussion}

\subsection{Band gaps}

\begin{table}
\begin{center}

\begin{tabular}{c|ccc}
			& this work & experiment & GGA \\
\hline
m-HfO$_2$ &	5.88 & 5.68~\cite{Balog1977} & \\
c-HfO$_2$ &	5.81 & & \\
t-HfO$_2$ &	!6.44! & & \\
am-HfO$_2$ & !4.06! & 5.72~\cite{Takeuchi2004} & \\

\end{tabular}

\end{center}
\end{table}

\subsection{Optical properties}

\begin{figure}
\begin{center}
	\includegraphics[width=\linewidth]{figures/c-HfO2.pdf}
	\caption{c-HfO$_2$ comparison with experiment}
\end{center}
\end{figure}

\begin{figure}
\begin{center}
	\includegraphics[width=\linewidth]{figures/m-HfO2.pdf}
	\caption{m-HfO$_2$ comparison with experiment}
\end{center}
\end{figure}


\section{Conclusions}

\bibliographystyle{unsrt}
\bibliography{bib-db}

\end{document}
