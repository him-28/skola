\documentclass[10pt,a4paper,twocolumn]{article}
\usepackage[english]{babel}
\usepackage[utf8]{inputenc}
\usepackage[plainpages=false,pdfpagelabels,unicode]{hyperref}
\usepackage[pdftex]{graphicx}
\usepackage[margin=1.5cm, includefoot]{geometry}
\usepackage{authblk}
\usepackage{xspace}
\usepackage[numbers,sort&compress]{natbib}

\author[a,b,c]{Pavel Ondračka}
\author[c]{David Holec}
\author[a,b]{Lenka Zajíčková}
\affil[a]{Faculty of Science, Masaryk University, Kotlářská 2, 611 37 Brno, Czech Republic}
\affil[b]{CEITEC - Central European Institute of Technology, Masaryk University, Kotlářská 2, 611 37 Brno, Czech Republic}
\affil[c]{Department of Physical Metallurgy and Materials Testing, Montanuniversität Leoben, Franz-Josef-Straße 18, Leoben A-8700, Austria}

\title{Optical properties of monoclinic, cubic, tetragonal and amorphous HfO2 with TB-mBJ}
\date{}

\begin{document}

\twocolumn[
  \begin{@twocolumnfalse}
    \maketitle
    \begin{abstract}    
    
    \end{abstract}
  \end{@twocolumnfalse}
]

\section{Introduction}
Something about HfO$_2$ polymorphs, transform temperatures etc, wide band gap etc...

HfO$_2$ is attracting a lot of attention as a high-$k$ material for electronic applications as well as optical applications such as antireflective coatings~\cite{Fadel1998, Khoshman2008}, heat-mirrors~\cite{Al-Kuhaili2004}, or laser mirrors~\cite{Meng2012}). 

On the ab initio front, HfO2 was focus of many theoretical studies for its structural, mechanical, electronic or optical properties \cite{Caravaca2005, Broqvist2007, Ceresoli2006, Garcia2004, Kaneta2007, Liu2009, Scopel2008, Terki2008, Zhao2002, Gruning2010}.

However, majority of the HfO$_2$ \textit{ab initio} studies employ conventional approximations for the exchange-correlation (xc) potential, local density approximation (LDA) and generalised gradient approximation (GGA). 
One shortcoming, of those approximations is the well known underestimation for the band gap.

%Since well calculated band gap and band structure is critical for correct prediction of optical properties. 

On the other hand, hybrid functionals or other more sophisticated methods as GW, give band gaps in much better agreement with experiment, however are much more computationally expansive.
A recently developed semi-local xc potential (modified Becke-Johnson, TB-mBJ) is an alternative that can provide highly accurate band gaps at the computational cost of LDA or GGA~\cite{Tran2009}.

Band gap of monoclinic HfO$_2$ was already found to be 5.83\,eV using the original TB-mBJ parameters~\cite{Koller2012}.
This is in a good agreement with the experimental value of .... \cite{}, it is however unclear, if this extends also to other forms of HfO$_2$.
Also the optical properties of HfO$_2$ calculated with TB-mBJ were to our knowledge not yet studied.
%This problem can be seen in case of TiO$_2$, where we have a very good agreement in calculated band gap for anatase, however the difference for calculated and experimental values for rutile is almost twice as big \cite{Sai2012}.  

%The huge impact of the improved description of the electronic structure on the predicted optical properties, such as refractive index, $n$, extinction coefficient, $\kappa$, or dielectric function $\epsilon$, when employing the mBJ potential was shown by \citet{Sai2012} for the case of rutile and anatase TiO$_2$.

In this work, we thoroughly test the applicability of TB-mBJ potential for monoclinic, tetragonal, cubic and amorphous HfO$_2$ with focus on band gap and optical properties.

\section{Methodology}

There is no associated exchange functional for TB-mBJ, hence its not suitable for structural optimization.
The initial HfO$_2$ monoclinic, tetragonal and cubic cells were structurally optimized with respect to internal positions and lattice parameters. 
This was done using the the Vienna Ab initio Simulation Package~\cite{Kresse1996}, with projector augmented pseudopotentials \cite{Kresse1999} and using the generalised gradient approximation parametrized by Perdew, Burke and Ernzerhof (GGA-PBE) \cite{Perdew1996}. 

Amorphous unit cell was prepared by a simulated annealing procedure: FIXME.

For determination of optical properties, we have taken the relaxed structures and used the Wien2k full potential all electron code~\cite{Blaha2001} with the linearized augmented plane wave method. The TB-mBJ was used for exchange and LDA for the correlation potential. For determination of dielectric functions we used the optic code~\cite{AmbroschDraxl2006}, a part of the Wien2k package.

Where comparing with room temperature experimental measurements, a Gaussian broadening of 0.03\,eV was used for the dielectric function. 

\section{Results a discussion}

\subsection{Structural properties}

For m-HfO$_2$ values of \citet{Adam1959} ($a = 5.1156$\,\r{A}, $b = 5.1722$\,\r{A}, $c = 5.2948$\,\r{A}, $\beta = 99.11^\circ$).

\subsection{Band gaps}

Table \ref{gaps} shows overview of calculated band gaps for different HfO$_2$ forms.
Those are compared to experiment and to band gaps calculated by other ab initio methods.
For m-HfO2 we have obtained a calculated band gap of of 5.75\,eV.
This is a perfect agreement with experimental value of 5.68\,eV~\cite{Balog1977} and comparable to HSE06 hybrid (5.98\,eV)~\cite{Komsa2010} and 5.9\,eV obtained by GW$_0$~\cite{Gruning2010}.
It is also in better agreement with experiment than PBE0 hybrid functional (6.75\,eV)~\cite{Komsa2010}.

Calculated band gap of c-HfO2 has a value of 5.88\,eV.
This seems to be a slight overestimation in comparison to hybrid functionals (SX: 5.6\,eV~\cite{Clark2010}, HSE06: 5.38\,eV~\cite{Yang2014}) and to GW$_0$ (5.5\,eV)~\cite{Gruning2010}.
Comparison to experiment is complicated, as cubic hafnia is not stable at room temperatures and hence it is usually stabilized by yttrium.
For (Y$_2$O$_3$)$_{15}$--(HfO$_2$)$_{85}$ the band gap is 5.8\,eV~\cite{Lim2002}, again comparable to our results.

TB-mBJ band gap for t-HfO$_2$ is 6.64\,eV, highest from all the hafnium polymorphs.
This is in agreement with the GW$_0$ calculation, where the band gap value of 6.0\,eV is also higher than band gap of monoclinic and cubic form, however the difference is lower.

Calculated optical band gap for amorphous hafnia is 5.65\,eV, again in good agreement with experimental values ranging from 5.49\,eV to5.72\,eV~\cite{Takeuchi2004, Nguyen2005}.
FIXME: finish calculations and include correct bad gap, also maybe include electronic band gap here for comparison with calculations.

All calculated band gaps are significantly improved in comparison to PBE values of 4.08, 3.77, and 4.79\,eV for monoclinic, cubic and tetragonal HfO2 respectively.
 

\begin{table*}
\begin{center}

\begin{tabular}{c|ccccc}
			& TB-mBJ & PBE & hybrid funkcionals & GW$_0$ & experiment \\
\hline
m-HfO$_2$ &	5.75 & 4.08 & PBE0: 6.75~\cite{Komsa2010}, HSE06: 5.98~\cite{Komsa2010} & 5.9~\cite{Gruning2010} & 5.68~\cite{Balog1977} \\
c-HfO$_2$ &	5.88 & 3.77 & SX: 5.6~\cite{Clark2010}, HSE06: 5.38~\cite{Yang2014} & 5.5~\cite{Gruning2010} & 5.8$^A$~\cite{Lim2002}\\
t-HfO$_2$ &	6.64 & 4.79 &  & 6.0~\cite{Gruning2010} & \\
am-HfO$_2$ & 5.53, 5.65 & & PBE0: 5.3~\cite{Broqvist2007}, 5.94~\cite{Chen2011} &  & 5.49--5.72~\cite{Takeuchi2004}, 5.62~\cite{Nguyen2005}\\

\end{tabular}
\caption{Overview of calculated band gaps compared to experiment and other ab initio calculations}
\label{gaps}
\end{center}
\end{table*}

\subsection{Optical properties}

Calculated optical properties are shown in 

\begin{figure}
\begin{center}
	\includegraphics[width=\linewidth]{figures/c-HfO2.pdf}
	\caption{c-HfO$_2$ comparison with experiment}
\end{center}
\end{figure}

\begin{figure}
\begin{center}
	\includegraphics[width=\linewidth]{figures/m-HfO2.pdf}
	\caption{m-HfO$_2$ comparison with experiment}
\end{center}
\end{figure}

\begin{figure}
\begin{center}
	\includegraphics[width=\linewidth]{figures/am-HfO2.pdf}
	\caption{am-HfO$_2$ comparison with experiment}
\end{center}
\end{figure}


\section{Conclusions}


\section*{Acknowledgments}
This work was supported by the IT4Innovations Centre of Excellence project (CZ.1.05/1.1.00/02.0070), funded by the European Regional Development Fund and the national budget of the Czech Republic via the Research and Development for Innovations Operational Programme, as well as Czech Ministry of Education, Youth and Sports via the project Large Research, Development and Innovations Infrastructures (LM2011033).

\bibliographystyle{unsrtnat}
\bibliography{bib-db}

\end{document}
