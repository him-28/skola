\documentclass[12pt]{article}
\usepackage[czech]{babel}
\usepackage[utf8]{inputenc}
\usepackage[plainpages=false,pdfpagelabels,unicode]{hyperref}
\usepackage[pdftex]{graphicx}
\usepackage[margin=2cm, includefoot]{geometry}
\usepackage{wrapfig}

\begin{document}

\title{Praktikum z patologické fyziologie \\
Experimentálně navozený diabetes
mellitus u pokusného zvířete - diagnostický
průkaz glukózovým tolerančním testem}
\author{Marie Ostrá}
\maketitle

\section{Úvod}

Diabetes mellitus (DM) je syndrom charakterizovaný deficiencí účinku inzulinu. Z hlediska
příčin deficitu inzulinu rozeznáváme dva základní typy onemocnění:
\begin{enumerate}
\item {DM typu I - porucha
vzniká v důsledku absolutního nedostatku inzulinu při autoimunitní destrukci $\beta$ buněk
Langerhansových ostrůvků pankreatu,} 
\item{DM typu II - relativní deficience vzniká v důsledku
inzulinové rezistence v periferních tkáních normálně citlivých na inzulin (kosterní sval, tuková
tkáň, játra).}
\end{enumerate}

\begin{wrapfigure}{r}{0.4\linewidth}
  \vspace{-20pt}
  \begin{center}
	\includegraphics[width=\linewidth]{Langerhanssche_Insel.jpg}
	\caption{Langerhansovy ostrůvky}
  \end{center}
  \vspace{-20pt}
\end{wrapfigure}

Při manifestním diabetu je možno zjistit hyperglykemii již nalačno ($\le$6.7mmol/l v žilní nebo
kapilární krvi po osmihodinovém lačnění). Její opakovaný průkaz stačí ke stanovení diagnózy
diabetu. Kromě manifestního diabetu ovšem existuje i tzv. porušená glukózová tolerance (PGT),
kdy je glykemie nalačno normální, ale po zátěži glukózou dlouho přetrvává hyperglykémie a její
maximální hodnota je vyšší než norma ($\le$11.1mmol/l za dvě hodiny po podání glukózy). K
rozlišení sporných případů diabetu a odhalení PGT slouží glukózový toleranční test. U člověka
jej v naprosté většině případů provádíme v modifikaci perorální, kdy vyšetřovaný vypije
standardní dávku rozpuštěné glukózy (75g) a glykémie se stanovuje v čase 0 (tj. nalačno), za 60 a
120 minut. U pokusného zvířete vzhledem k nesnadné perorální aplikaci glukózy použijeme
modifikaci intraperitoneální. Krev na stanovení glykémie odebíráme z ocasní žíly v čase 0, 30 a
90 minut.

K vyvolání experimentálního diabetu se u zvířat nejčastěji používá alloxan nebo streptozotocin.
Obě látky selektivně toxicky poškozují $\beta$ buňky Langerhansových ostrůvků pankreatu a vedou v
závislosti na dávce k rozvoji různě závažné inzulinopenie, a tedy diabetu typu I. Působení
alloxanu podaného intravenózně v dostatečně vysoké dávce je experimentálně dobře popsáno.
Alloxan vyvolá nejprve přechodnou stimulaci $\beta$ buněk, při které dojde k uvolnění intracelulárních
zásob inzulinu a vznikne krátkodobá hypoglykemie. Po několika dnech se pak u zvířat objeví
inzulinopenie s klasickými příznaky diabetu: hyperglykémie, glykosurie, polydipsie a polyurie.
Tíže projevů je závislá na použité dávce: zničení všech $\beta$ buněk, a tedy kompletní inzulinopenie,
je dosaženo dávkou cca 65mg/kg váhy zvířete (potkan), v našem případě volíme záměrně dávku
nižší.

Jako průkaz rozvoje DM u potkana provedeme modifikovaný glukózový toleranční test -
stanovíme glykémii nalačno a za 30 a 90 min po intraperitoneální aplikaci 20\% glukózy.

\begin{figure}
	\begin{centering}
	\includegraphics[width=0.6\linewidth]{DM.jpg}
	\caption{Ukázka hodnocení orálního glukózového tolerančního testu u člověka (plná kapilární,
popř. žilní krev)}
	\end{centering}
\end{figure}


\section{Postup}
\subsection{Experimentální navození diabetu}
Zvíře uvedeme do anestézie (inhalační úvod, anestetická směs 0.5ml/100g i.p.) a fixujeme na
operačním stolku v hřbetní poloze. Nůžkami provedeme opatrně incizi kůže na krku ve stř. čáře.
Zpřístupníme operační pole v místě, kde se v. jugularis externa zanořuje pod musculus pectoralis
a kanylujeme v. jugularis. Cévu nepreparujeme ani jinak netraumatizujeme, intradermální jehlu
zavedeme kraniálně přes m. pectoralis a bez aspirace aplikujeme roztok alloxanu v objemu
0.1ml/100g váhy zvířete. U kontrolních zvířat stejným způsobem aplikujeme fyziologický roztok.
Operační ránu uzavřeme několika stehy. Zvířeti v mezidobí poskytneme náležitou pooperační
péči a nutrici, zejm. však dostatečnou hydrataci s ohledem na rozvíjející se hyperglykémii. Další
fáze pokusu následuje cca za 5 dní.

\subsection{Glukózový toleranční test}
Zvířata obvyklým způsobem uvedeme do narkózy. Okamžitě po nastoupení plné anestézie
vyšetříme glykémii nalačno (zvířata jsou v den pokusu lačná). Glykémii stanovíme ze vzorku
žilní krve pomocí osobního glukometru dle doporučení výrobce. Potkanovi otřeme proximální
část ocasu tamponem navlhčeným v alkoholu. Žiletkou nařízneme příčně kůži v místech, kde
probíhá ocasní žíla. Kapku krve opatrně obtiskneme na reagenční zónu proužku. Kapka musí být
dostatečně velká, ale nesmí přetéct přes okraje reagenční zóny. Krvácení zastavíme kompresí
tamponem. Po změření glykémie nalačno zvířeti aplikujeme 20\% glukózu i.p. v dávce 1ml/100g
(tj. 2g/kg). Stejným způsobem změříme glykémii za 30 a 90 minut po podání. V mezičasech mezi
měřeními položíme zvíře na břicho a pod pánev vložíme Petriho misku na zachycení moči. Na
závěr praktika stanovíme pomocí reagenčních proužků semikvantitativně přítomnost glukózy a
ketolátek v zachycené moči.

\section{Výsledky}

\begin{table}[bh]
\begin{tabular}{|c|c|c|c|c|c|c|c|c|c|c|}
\hline
Rank Sum & Rank Sum & $U$ & $Z$ & $p$-value & $Z$ & $p$-value & Valid & Valid & 2*1sided \\
kont & DM & & & & adjusted & & $N$ kont & $N$ DM & exact $p$ \\
\hline
898,5 & 1516,500 & 388,5 & 1,648 & 0,0993 & 1,6484 & 0,0993 & 22 & 47 & 0,0983 \\
\hline
\end{tabular}
\caption{Mann-Whitney $U$ Test pro glykemii v čase 0 minut. Marked tests are significant at $p < 0,05$}
\end{table}

\begin{figure}[bh]
	\begin{centering}
	\includegraphics[width=0.75\linewidth]{glykemie0-box.pdf}
	\caption{Glykémie v čase 0 minut}
	\end{centering}
\end{figure}

\begin{table}
\begin{tabular}{|c|c|c|c|c|c|c|c|c|c|c|}
\hline
Rank Sum & Rank Sum & $U$ & $Z$ & $p$-value & $Z$ & $p$-value & Valid & Valid & 2*1sided \\
kont & DM & & & & adjusted & & $N$ kont & $N$ DM & exact $p$ \\
\hline
669 & 1746 & 416 & 1,294 & 0,196 & 1,294 & 0,196 & 22 & 47 & 0,197 \\
\hline
\end{tabular}
\caption{Mann-Whitney $U$ Test pro glykemii v čase 60 minut. Marked tests are significant at $p < 0,05$}
\end{table}

\begin{figure}
	\begin{centering}
	\includegraphics[width=0.75\linewidth]{glykemie30-box.pdf}
	\caption{Glykémie v čase 30 minut}
	\end{centering}
\end{figure}

\begin{table}
\begin{tabular}{|c|c|c|c|c|c|c|c|c|c|c|}
\hline
Rank Sum & Rank Sum & $U$ & $Z$ & $p$-value & $Z$ & $p$-value & Valid & Valid & 2*1sided \\
kont & DM & & & & adjusted & & $N$ kont & $N$ DM & exact $p$ \\
\hline
409 & 2006 & 156 & -4,642 & 3e-6 & -4,647 & 3e-6 & 22 & 47 & 1e-6 \\
\hline
\end{tabular}
\caption{Mann-Whitney $U$ Test pro glykemii v čase 90 minut. Marked tests are significant at $p < 0,05$}
\end{table}

\begin{figure}
	\begin{centering}
	\includegraphics[width=0.75\linewidth]{glykemie90-box.pdf}
	\caption{Glykémie v čase 90 minut}
	\end{centering}
\end{figure}

\clearpage
\section{Závěr}
Praktikum proběhlo úspěšně. U laboratorního potkana se podařilo indukovat diabetes melitus podáním roztoku alloxanu. Po aplikaci glukózy se glykemie u zvířat s indukovaným diabetem zvýšila a s postupujícím časem u dále narůstala. Toto je v přímém kontrastu s pozorováními na kontrolních potkanech, kde po podání glukózy sice dojde prvně k nárustu glykemie, která nicméně po čase zase poklesne. V moči byl u diabetických zvířat pozorovnán výskyt glukózy.

\end{document}
