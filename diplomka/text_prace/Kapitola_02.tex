\chapter{Model dielektrické odezvy DLC vrstev}
Greek test
$\alpha \beta \gamma \delta \chi \epsilon \kappa \lambda \mu \nu \omega \pi \rho \psi \zeta$

$\alpha \beta \Gamma \Delta \chi \epsilon \kappa \Lambda \mu \nu \Omega \Pi \rho \Psi \zeta$
FIXME - úvod + všechno co potřebujeme 

KKintegrál, time-reversal symetry a zobecněné f-sum pravidlo.

První podmínka, kterou musí vhodný model lineární dielektrické odezvy splňovat, je Kramers--Kronigův integrál (citovat KK původní články??? nebo nějakou učebnici???), který spojuje reálnou a imaginární část dielektrické funkce  
\begin{equation}
\epsilon_\mathrm{r}(\omega) = 1 + \frac{1}{\pi} \int (přeškrtnutý) \frac{\epsilon_\mathrm{i}(\zeta)}{\zeta - \omega} \mathrm{d}\zeta \mathrm{,}
\label{KKint}
\end{equation}
kde $\omega$ je úhlová frekvence a $\epsilon_\mathrm{r}, \epsilon_\mathrm{i}$ jsou reálná a imeginární část dielektrické funkce.  

\begin{equation}
\epsilon(\omega) =\epsilon^* (-\omega) \mathrm{,}
\label{casovasymetrie}
\end{equation}

Třetí podmínka, pro model dielektrické odezvy, je takzvané f-sumační pravidlo
\begin{equation}
\int_0^\infty \epsilon_\mathrm{i} (\omega) \omega \mathrm{d} \omega = \frac{\pi}{2} \omega_\mathrm{p}^2 = \frac{\pi}{2} \frac{e^2 n_\mathrm{e}}{ \epsilon_0 m_\mathrm{e}} \mathrm{,}
\end{equation}
kde $\omega_\mathrm{p}$ je plazmová frekvence, $n_\mathrm{e}$ je koncentrace elektronů, $e$ je elementární náboj a $m_\mathrm{e}$ je hmotnost elektronu. F-sumační pravidlo je důsledek kombinace Kramers-Kronigova integrálu s Newtonovými zákony pro nerelativistickou (klasickou???) nabitou částici a spojuje hustotu náboje s dielektrickou funkcí. Klasické f-sumační pravidlo můžeme chápat jako integrální formu Thomas--Reiche--Kuhnova sumačního pravidlo z kvantové mechaniky, pokud ho aplikujeme na intrakci světla s pevnou látkou.  
 

\section{ Parametrizace funkce dielektrické  odezvy}
Dielektrická funkce je seskládána z jednotlivých částí, které reprezentují jednotlivé typy přechodů $t$. Celková dielektrická funkce je suma přes dielektrické funkce jednotlivých částí a vakua \cite{sumrule1}. 
\begin{equation}
\epsilon(E) = 1 + \sum_t \epsilon_t(E)
\label{suma1}
\end{equation}

Dále zavedem takzvanou "funkci síly přechodu (transition strength function)"? $F(E)$ která se skládá z funkci síly přechodu pro jednotlivé říspěvky t.
\begin{equation}
F(E) = \sum_t F_t(E)
\end{equation}
Je praktické jednotlivé příspěvky $F_t$ normalizovat na $F_t^0$ aby platilo
\begin{equation}
\int_0^\infty F_t^0(E)\mathrm{d}E = 1
\end{equation}

Integrováním přes $F_t(E)$ získáme 
\begin{equation}
\sum_t \int_0^\infty F_t(E)\mathrm{d}E = \sum_t N_t = N \mathrm{.}
\end{equation}
$N_t$ jsou integrální síly přechodů pro jednotlivé příspěvky a $N$ je celková integrální síla přechodu, nebo jen jednoduše celková síla přechodu.  

Podobně můžeme zavést takzvanou normalizovanou dielektrickou funkci $\epsilon_t^0$ pro jednotlivé příspěvky. Platí 
\begin{equation}
\epsilon_t(E) = N_t \epsilon_t^0  \, \mathrm{,}
\end{equation}
kde $N_t$ je takzvaná integrální síla přechodu pro jednotlivé příspěvky.

Pak můžeme přepsat rovnici (\ref{suma1}) jako:
\begin{equation}
\epsilon (E) = 1 + \sum_t N_t \epsilon_t^0(E) \mathrm{.}
\end{equation}

\section{Mezipásové přechody mezi valenčním a vodivostním pásem}
V amorfním materiálu s jedním valenčním a vodivostním pásem je příspěvek k dielektrické funkci nenulový v intervalu energií od $E_\mathrm{g}$ do $E_\mathrm{h}$, kde $E_\mathrm{g}$ je šířka zakázaného pásu a $E_\mathrm{h}$ je rozdíl energií mezi minimem valenčního a maximem vodivostního pásu (maximální energie přechodu mezi valenčním a vodivostním pásem). Následující parametrizace se ukázala jak vhodná. FIXME!

\begin{equation}
\epsilon_\mathrm{i,vc}^0 = \mathrm{sgn}(E) \frac{(|E|- E_\mathrm{g})^2(E_\mathrm{h} - |E|)^2}{ C E^2 [(|E| - E_\mathrm{c})^2 + B_\mathrm{c}^2]} \Pi_{E_\mathrm{g},E_\mathrm{h}}(|E|) \, \mathrm{,}
\end{equation}
kde $E_\mathrm{c}$ a $B_\mathrm{c}$ jsou parametry zahrnující rozšíření pásové struktury a její nesymetrii vzhledem k Fermiho energii?????

Funkce $\Pi_{E_\mathrm{min},E_\mathrm{max}}(|E|)$ je definována následovně:
\begin{equation}
\Pi_{E_\mathrm{min},E_\mathrm{max}}(|E|) = 1 : E_\mathrm{min} < E < E_\mathrm{max}
 FIXME!!!!! 0 : \mathrm{jinak.}
\end{equation}




\cleardoublepage
