\chapter{Diskuse výsledků}
\def\floatpagefraction{0.85}

V rámci této práce bylo celkem naměřeno a~zkoumáno osm vrstev. Pro správné vyhodnocení celkové vodíkové koncentrace bylo nejprve potřeba určit relativní síly přechodu $\alpha_j$ pro jednotlivé vibrační módy CH skupin. Původní idea bylo použít teoreticky vypočítané hodnoty z~odborné literatury. Žádnou odpovídající teorii se ale bohužel nepodařilo nalézt. Hodně publikací nicméně obsahuje experimentálně určené efektivní dynamické náboje $e_j^*$ pro CH skupiny, které se dají přepočítat na relativní síly přechodu jako \cite{sumrule2}
%
\begin{equation}
\frac{e_j^*}{e} = Z_j \sqrt{\alpha_j} \text{.}
\label{efch2str}
\end{equation}
%
Přehled více takových hodnot lze nalézt například v~\cite{Heitz1998}. Bohužel se tyto studie omezují na jednoduché uhlovodíky a~polymerní a-C:H vrstvy. Jejich hodnoty proto nedávaly dobré výsledky při mých pokusech o jejich přímou aplikaci na fitování DLC vrstev. 

Relativní síly přechodu byly pro jednotlivé CH skupiny nakonec určeny pomocí multi\-sample fitu čtyř vrstev. Princip spočíval v~tom, že celková koncentrace vodíku byla pro všechny čtyři vrstvy zafixována na hodnoty určené z~ERDA. Volnými parametry pro jednotlivé vrstvy byly pak parciální koncentrace vodíku v~jednotlivých skupinách, které nicméně dohromady musejí dávat celkovou vodíkovou koncentraci. Klíčové parametry sil přechodu byly taktéž volné parametry, ale byly pro všechny vrstvy spjaty na stejné hodnoty. 

Další parametry, které byly pro všechny vrstvy stejné jsou polohy píků uhlovodíkových skupin, ty byly zafixovány na hodnoty získané z~literatury \cite{Robertson2002, Dischler1983, Ristein1998, Zajickova2011}, podrobný popis jednotlivých píků bude proveden níže. Ostatní parametry, například pološířky píků, byly pro jednotlivé vrstvy samostatné.

Kromě zafixování poloh absorpčních píků a~sepnutí sil přechodu pro všechny vrstvy dohromady bylo při fitování předpokládáno, že
u skupin, které mají symetrické i~antisymetrické valenční vibrace, je antisymetrický pík vždy silnější a~dále  že symetrický i~antisymetrický pík stejné skupiny mají stejnou pološířku \cite{Heitz1998}. V souladu s~hodnotami pološířek píků z~jiných prací \cite{Dischler1983, Zajickova2011} byly rovněž kladeny omezení na maximální a~mi\-ni\-mální šířky píků.

Kalibrovány byly píky valenčních vibrací sp$^2$CH$_{1,2}$ a~sp$^3$CH$_{1,2,3}$ v~oblasti 2800--3100\,cm$^{-1}$ a~kolébavé vibrace sp$^3$CH$_3$ na 1375\,cm$^{-1}$. Ve spektru jsou další vodíkové píky, například píky kolébavých vibrací kolem 1450\,cm$^{-1}$, ale analýza je komplikovaná tím, že v~této oblasti je křemík absorbující. Podrobný rozbor vlivu substrátu bude proveden později.

Z důvodu absence experimentálních dat nad 6\,eV bylo také potřeba zafixovat některé parametry elektronové struktury, především pro $\sigma$ elektrony, parametry přechodů do vyšších excitovaných stavů a parametry pro excitace jaderných elektronů. K tomuto účelu byly použity hodnoty získané z~předchozí práce na vrstvě CH30A, ve které byly optické konstanty a~elektronová struktura zkoumány pomocí synchrotronové elipsometrie v~rozsahu až do 30\,eV \cite{Franta2011}.

 Podařilo se dosáhnout velmi dobré shody mezi experimentálními daty a~fitovanými funkcemi. Srovnání měřených dat s~fitem je na na obrázcích \ref{R-multifit} a~\ref{R-multifit2} pro odrazivost, a~na obrázcích \ref{ell-multifit} a~\ref{ell-multifit2} pro elipsometrii ve viditelné a~blízké ultrafialové oblasti.  Srovnání měřených dat s~fitem pro propustnost v~infračervené oblasti je na obrázcích \ref{T-multifit} a~\ref{T-multifit2}. Parametry $\chi$ charakterizující kvalitu fitu jsou v~tabulce \ref{fitparams}. Diskuse jednotlivých parametrů dále pokračuje v~následujících podkapitolách.

\begin{table}[tbhp]
 \centering
	\renewcommand{\tabcolsep}{4pt}
 \begin{tabular}{lcccccccc}
\hline
parametry & \multicolumn{4}{c}{Kalibrační multifit} & \multicolumn{4}{c}{Samostatné fity}\\
 & CH30A & CH83A & CH88A & CH90A & CH30C & CH30D & CH87A & CH89A\\
\hline
\multicolumn{9}{l}{DLC vrstva}\\
\hline
$N_\mathrm{a}$\,[eV$^2$]$^\mathrm{a}$ & 364 & 364 & 364 & 364 & 364 & 364 & 364 & 364\\
$f_\mathrm{Csp3}$ & 0.320 & 0.416 & 0.430 & 0.362 & 0.304 & 0.285 & 0.391 & 0.508\\
$f_\mathrm{H}$ ERDA & 0.326 & 0.480 & 0.380 & 0.310 & 0.306 & 0.303 & 0.480 & 0.350\\
$f_\mathrm{H}$ & 0.326 & 0.480 & 0.380 & 0.310 & 0.315 & 0.322 & 0.445 & 0.399\\
$f_\mathrm{H_{sp3CH3}}$ & 0.036 & 0.123 & 0.085 & 0.079 & 0.035 & 0.051 & 0.113 & 0.075\\
$f_\mathrm{H_{sp3CH2}}$ & 0.094 & 0.160 & 0.158 & 0.086 & 0.061 & 0.062 & 0.171 & 0.156\\
$f_\mathrm{H_{sp3CH}}$ & 0.134 & 0.043 & 0.040 & 0.062 & 0.170 & 0.176 & 0.044 & 0.038\\
$f_\mathrm{H_{sp2CH2}}$ & 0.000 & 0.082 & 0.005 & 0.000 & 0.000 & 0.000 & 0.033 & 0.057\\
$f_\mathrm{H_{sp2CH}}$ & 0.062 & 0.072 & 0.093 & 0.083 & 0.049 & 0.033 & 0.084 & 0.072\\
$d_\mathrm{fE}$\,[nm] & 108 & 574 & 457 & 115 & 111 & 114 & 394 & 446\\
$d_\mathrm{fR}$\,[nm] & 108 & 571 & 457 & 115 & 111 & 114 & 393 & 444\\
$d_\mathrm{fM}$\,[nm] & 108 & 571 & 455 & 115 & 111 & 114 & 392 & 440\\
$E_\mathrm{g\pi}$\,[eV] & 1.2 & 1.4 & 1.1 & 1.1 & 1.2 & 1.1 & 1.4 & 1.1\\
$E_\mathrm{h\pi}$\,[eV]$^\mathrm{b}$ & 20.4 & 20.4 & 20.4 & 20.4 & 20.4 & 20.4 & 20.4 & 20.4\\
$E_\mathrm{c\pi}$\,[eV] & 4.8 & 2.5 & 3.7 & 4.2 & 4.9 & 5.1 & 5.0 & 2.2\\
$B_\mathrm{c\pi}$\,[eV] & 2.6 & 2.6 & 1.8 & 2.5 & 2.4 & 2.4 & 4.3 & 2.3\\
$E_\mathrm{g\sigma}$\,[eV] & 4.4 & 2.2 & 3.0 & 3.8 & 4.8 & 5.4 & 2.2 & 2.0\\
$E_\mathrm{h\sigma}$\,[eV]$^\mathrm{b}$ & 40.1 & 40.1 & 40.1 & 40.1 & 40.1 & 40.1 & 40.1 & 40.1\\
$E_\mathrm{c\sigma}$\,[eV]$^\mathrm{b}$ & 10.9 & 10.9 & 10.9 & 10.9 & 10.9 & 10.9 & 10.9 & 10.9\\
$B_\mathrm{c\sigma}$\,[eV]$^\mathrm{b}$ & 8.8 & 8.8 & 8.8 & 8.8 & 8.8 & 8.8 & 8.8 & 8.8\\
$\alpha_\mathrm{K}$$^\mathrm{b}$ & 2 & 2 & 2 & 2 & 2 & 2 & 2 & 2\\
$E_\mathrm{K}$\,[eV]$^\mathrm{b}$ & 284 & 284 & 284 & 284 & 284 & 284 & 284 & 284\\
\multicolumn{9}{l}{}\\
\multicolumn{9}{l}{Přechodová vrstva}\\
\hline
$Q_\mathrm{t}$\,[eV$^{3/2}$]$^\mathrm{a}$ & 13.7 & 13.7 & 13.7 & 13.7 & 13.7 & 13.7 & 13.7 & 13.7\\
$E_\mathrm{g}$\,[eV]$^\mathrm{a}$ & 1.4 & 1.4 & 1.4 & 1.4 & 1.4 & 1.4 & 1.4 & 1.4\\
$E_\mathrm{h}$\,[eV]$^\mathrm{a}$ & 5.2 & 5.2 & 5.2 & 5.2 & 5.2 & 5.2 & 5.2 & 5.2\\
$Q_\mathrm{G}$\,[eV$^{3/2}$]$^\mathrm{a}$ & 2.8 & 2.8 & 2.8 & 2.8 & 2.8 & 2.8 & 2.8 & 2.8\\
$E_\mathrm{G}$\,[eV]$^\mathrm{a}$ & 1.8 & 1.8 & 1.8 & 1.8 & 1.8 & 1.8 & 1.8 & 1.8\\
$E_\mathrm{G}$\,[eV]$^\mathrm{a}$ & 0.5 & 0.5 & 0.5 & 0.5 & 0.5 & 0.5 & 0.5 & 0.5\\
$d_\mathrm{t}$\,[nm] & 3.1 & 40.9 & 23.0 & 3.2 & 3.5 & 4.7 & 39.3 & 10.3\\
\multicolumn{9}{l}{}\\
\multicolumn{9}{l}{Substrát (Si)}\\
\hline
$f_\mathrm{O}$\,[10$^{-6}$]$^\mathrm{c}$ & - & - & - & - & 22.80 & 22.58 & 20.77 & 20.94\\
$d_\mathrm{s}$\,[$\mu$m] & 327 & 318 & 314 & 328 & 353 & 370 & 381 & 374\\
\multicolumn{9}{l}{}\\
\multicolumn{9}{l}{Vrstva na zadní straně (SiO$_2$)}\\
\hline
$d_\mathrm{b}$\,[nm] & 1.4 & 1.2 & 2.0 & 4.3 & 1.3 & 1.5 & 0.8 & 1.4\\
\hline
\multicolumn{9}{l}{$^\mathrm{a}$ parametr byl volný pouze při kalibračním fitu a byl společný pro všechny vrstvy}\\
\multicolumn{9}{l}{$^\mathrm{b}$ parametr byl zafixován na hodnoty z \cite{Franta2011}}\\
\multicolumn{9}{l}{$^\mathrm{c}$ pro kalibrační multifit byly použity tabulkové hodnoty}\\
\multicolumn{9}{l}{FIXME}\\
\end{tabular}

 \caption{Parametry DLC vrstev.}
\label{fitparams}
\end{table}

\begin{figure}[htp]
	\includegraphics[width=\linewidth]{grafy/R-multifit.pdf}
	\caption{Porovnání naměřené a~fitované odrazivosti vzorků vzorků CH30A, CH83A, CH88A a~CH90A ve viditelné a~blízké ultrafialové oblasti.}
	\label{R-multifit}
\end{figure}

\begin{figure}[htp]
	\includegraphics[width=\linewidth]{grafy/R-multifit2.pdf}
	\caption{Porovnání naměřené a~fitované odrazivosti vzorků CH30C, CH30D, CH87A a~CH89A ve viditelné a~blízké ultrafialové oblasti.}
	\label{R-multifit2}
\end{figure}

\begin{figure}[htp]
	\includegraphics[width=\linewidth]{grafy/ell-multifit.pdf}
	\caption{Porovnání naměřeného a~fitovaného přidruženého elipsometrického parametru Ic$_{\mathrm{II}}$ pro vzorky CH30A, CH83A, CH88A a~CH90A pro úhel dopadu 70$^\circ$.}
	\label{ell-multifit}
\end{figure}

\begin{figure}[htp]
	\includegraphics[width=\linewidth]{grafy/ell-multifit2.pdf}
	\caption{Porovnání naměřeného a~fitovaného přidruženého elipsometrického parametru Ic$_{\mathrm{II}}$ pro vzorky CH30A, CH83A, CH88A a~CH90A pro úhel dopadu 70$^\circ$.}
	\label{ell-multifit2}
\end{figure}

\section{Výsledky z~RBS/ERDA}
Měření hustot vrstev a~koncentrací vodíku, pomocí kterých byla provedena kalibrace a~kontrola optických metod provedli RNDr. Vratislav Peřina CSc. a~Mgr. Romana Mikšová z~Ústavu jaderné fyziky Akademie Věd České Republiky v~Řeži. K měření byly použity metody RBS a~ERDA. Metoda RBS, která dobře funguje pro těžší prvky, byla použita k~měření koncentrace uhlíku ve vrstvách a~celkové hustoty, pro lehčí vodík pak byla použita metoda ERDA. Výsledky jsou uvedeny v~tabulce \ref{ERDA-table}. Pro vrstvu CH30A a~žíhané vrstvy CH30C a~CH30D byla rovněž měřena koncentrace kyslíku $f_\mathrm{O}$, pro ostatní vrstvy byl detekován pouze vodík a~uhlík.

\begin{table}[h!]
 \centering
	\renewcommand{\tabcolsep}{4pt}
 \begin{tabular}{lcccccccc}
\hline
 & CH30A & CH30C & CH30D & CH83A & CH87A & CH88A & CH89A & CH90A\\
\hline

$f_\mathrm{H}$ & 0,326 & 0,306 & 0,303 & 0,48 & 0,48 & 0,38 & 0,35 & 0,31\\
$f_\mathrm{C}$ & 0,654 & 0,673 & 0,662 & 0,52 & 0,52 & 0,62 & 0,65 & 0,69\\
$f_\mathrm{O}$ & 0,02  & 0,021 & 0,035 & -    & -    &  -   & -    & -   \\ 

$\rho$\,[g/cm$^3$] & 1,77 & 1,68 & 1,76 & 1,31 & 1,4 & 1,6 & 1,65 & 1,45\\
$N_\mathrm{a}$\,[eV$^2$] & 274 & 254 & 265 & 254 & 272 & 267 & 264 & 220 \\
\hline

\end{tabular}

 \caption{Složení a~hustoty vrstev získané z~RBS/ERDA, atomární hustota $N_\mathrm{a}$ byla dopočítána podle vztahu (\ref{mdensity}).}
\label{ERDA-table}
\end{table}

Zde je nutno podotknout, že podobně jako u optických měření, i~metody analýzy povrchů pomocí nabitých částic jsou velmi závislé na správné analýze dat. Vyhodnocování totiž podobně jako u optiky probíhá fitováním simulovaného a~měřeného spektra. Proto třeba pro vzorek CH90A byly podle způsobu analýzy dat získány pro koncentraci vodíku postupně hodnoty 23,5\,\%, 36,3\,\% a~31\,\%. U ostatních vzorků k~takovým výrazným výkyvům nedocházelo, nicméně stejně musíme k~výsledkům získaným z~RBS/ERDA přistupovat kriticky a~to i~proto, že odhadnutá chyba maření je okolo 4\,%.

\section{Parametry elektronové struktury, poměr sp$^2$ a~sp$^3$ uhlíků a~hustota}
Parametry elektronové struktury u DLC vrstev ovlivňují hlavně odrazivost, propustnost a~elipsometrické veličiny ve viditelné a~ultrafialové oblasti a~na IR oblast nemají velký vliv. 
Proto se většina prací, které zkoumají koncentraci vodíku v~DLC vrstvách pomocí optických měření, zabývá pouze infračervenou oblastí. Tato práce se nicméně věnuje i~viditelné a~blízké ultrafialové oblasti, protože ty poskytují zajímavé údaje o struktuře látky, které nám dále mohou pomoci při úvahách o vodíkové koncentraci a~hlavně na nich velmi závisí atomární hustota $N_\mathrm{a}$, kterou musíme dobře určit, protože relativní koncentrace vodíku je na ní přímo závislá. 

Hodnoty parametrů elektronové struktury, hodnoty parametrů vodíkové koncentrace včetně rozložení vodíku mezi jednotlivé CH skupiny a~další zjištěné hodnoty parametrů DLC vrstev systémů jsou shrnuty v~tabulce \ref{fitparams}. Vzhledem k~počtu parametrů, počtu vrstev a~velikosti tabulek nejsou v~tabulkách uváděny přesné hodnoty chyb fitovaných veličin, nicméně veličiny jsou zaokrouhleny tak, že chyba (na intervalu spolehlivosti 1$\sigma$) je vždy na posledním platném místě. Hodnoty tloušťky vrstev $d_\mathrm{f}$ byly pro každé měření samostatné, $d_\mathrm{fR}$ je tloušťka pro odrazivost, $d_\mathrm{fE}$ je tloušťka pro elipsometrii a~$d_\mathrm{fT}$ je tloušťka pro propustnost. Toto rozdělení bylo děláno proto, že přestože jsou vzorky poměrně malé, ne vždy se podařilo měřit při všech měřeních stejné místo vzorku. Nicméně z~určených hodnot vidíme, že odchylky jsou minimální. Pro nejtenčí vrstvy byla tloušťka pro IR propustnost $d_\mathrm{fT}$ spjata s~tloušťkou pro elipsometrii $d_\mathrm{fE}$, protože pro vrstvy o tloušťce okolo 100\,nm již v~IR oblasti téměř nedochází k~interferencím a~při určování tloušťky by proto mohlo dojít k~větším chybám. 

Začněme diskusí šířky zakázaného pásu $\pi$ elektronů, $E_\mathrm{g\pi}$ je pro všechny vrstvy v~očekávaném rozmezí 1--1,5\,eV, to odpovídá typickým hodnotám pro DLC vrstvy \cite{Demichelis1992, Franta2007}. 
Zajímavý je posun šířky zakázaného pásu $\sigma$ elektronů pro některé vrstvy na velmi malé hodnoty okolo 2\,eV, přičemž jiné práce udávají většinou hodnoty 4--5\,eV \cite{Demichelis1992, Franta2011}. Toto nastává hlavně pro tlusté vrstvy CH83A, CH87A, CH89A, mírně pro CH88A. Možné vysvětlení je nehomogenita vrstev. Nehomogenní jsou všechny vrstvy a~je to přímý důsledek de\-po\-zi\-ce energetickým bombardováním. 
Z toho důvodu bychom očekávali na povrchu vrstvy nižší hustotu než při substrátu, kde by naopak hustota měla být nejvyšší. Podobně poměr $f_\mathrm{Csp3}/f_\mathrm{Csp2}$ by měl být také na povrchu vrstvy nejnižší. Nehomogenitu do modelu zahrnout lze a~vede to k~výrazně lepším výsledkům \cite{Franta2011}, nicméně také dochází k~velmi výraznému zpomalení fitování kvůli nutnosti numerických výpočtů. 

Nehomogenita tak nakonec nebyla do modelu zahrnuta a~to především proto, že hlavní zaměření této práce bylo na IR oblast, kde se nehomogenita tak výrazně neprojevuje. Dalším důvodem také bylo, že použitý model je už i~tak značně komplikovaný a~to především pro kalibrační multisample fit, kde bylo dohromady přes sto volných parametrů. 

Neblahým důsledkem tohoto zjednodušení je pak to, že pro tlustší vrstvy, kde se nehomogenita projevuje nejvíc, nedostáváme úplně korektní výsledky právě pro poměr $f_\mathrm{Csp3}/f_\mathrm{Csp2}$. Konkrétně pro vzorky CH89A a~CH83A dostáváme hodnoty 4 a~5,3, přičemž bychom očekávali hodnoty kolem 1 \cite{Robertson2002}. 
Z toho pak samozřejmě plynou výše uvedené změny v~určených parametrech elektronové struktury, které se snaží kompenzovat odchylky vzniklé právě absencí nehomogenity v~modelu. Toto se potvrzuje na vrstvě CH88A, která byla deponována ne najednou, ale s~přestávkami, aby došlo k~chladnutí vrstvy během depo\-zi\-ce a~omezení nehomogenity. 
U této vrstvy sice také můžeme vidět, že poměr $f_\mathrm{Csp3}/f_\mathrm{Csp2}$ je 2,2, což se stále liší od hodnot okolo jedničky u tenčích vrstev, ale nejedná se o takový rozdíl jako u jiných tlustých vrstev. 
U žíhaných vrstev pak můžeme vidět pokles od 0.9 pro vzorek CH30A, přes 0.8 pro vzorek CH30C až po 0.72 pro vzorek CH30C. To je důsledkem grafitizace vrstev v~důsledku zahřívání. Je tedy vidět, že teplotní stabilita vrstev není nijak velká. 

Co se týče celkového indexu lomu, ten je pro všechny vrstvy znázorněn na obrázku \ref{index}. Pro DLC vrstvy se někdy udává experimentální vztah mezi indexem lomu a~hustotou vrstvy, který říká, že pro DLC vrstvy je čtverec indexu lomu na 633\,nm přímo úměrný hustotě vrstvy \cite{Donnet2008}. Toto ne úplně dobře odpovídá výsledkům z~této práce, protože podle ERDA mají největší hustotu vrstvy CH30A, CH30C a~CH30D, které ale mají ze všech vrstev index lomu nejnižší. 

\begin{figure}[tbhp]
	\includegraphics[width=\linewidth]{grafy/N.pdf}
	\caption{Spektrální závislost indexu lomu pro všechny vrstvy.} 
	\label{index}
\end{figure}

Tyto problémy přisuzuji kromě nehomogenity také tomu, že v~našem modelu uvažujeme samostatně přechodovou vrstvu a~vrstvu, zatímco při určování výše uvedené experimentální závislosti nebyla přechodová vrstva uvažována. Z~toho důvodu nejsou uvedené indexy lomu z~\cite{Donnet2008} přímo indexy lomu DLC vrstev, ale efektivními indexy lomu dvojvrstev přechodová vrstva -- vrstva. Další možné vysvětlení je takové, že rozdíly mezi jednotlivými vrstvami nejsou u indexu lomu tak velké, aby bylo možno tento experimentální poznatek o závislosti indexu lomu a~hustoty uplatnit. 

Výše uvedený nesoulad hodnot indexu lomu s~hustotou určenou s~ERDA/RBS naznačuje, že hustotám určeným právě pomocí ERDA/RBS nelze slepě důvěřovat. S hustotou určenou z~optických měření je ale problém takový, že závisí velmi výrazně na dielektrické funkci pro vyšší energie, integrujeme přes všechny energie $\epsilon_i E \mathrm{d}E$, viz vztah (\ref{density}). 
Jak bylo již uvedeno, máme k~dispozici parametry elektronové struktury pro vyšší energie určené v~dřívějších pracích a~původní předpoklad byl ten, že parametry elektronové struktury se mezi vrstvami moc neliší a~vrstvy se liší především v~hustotě a~poměru sp$^2$ a~sp$^3$ uhlíku. Když se ale bohužel ukázalo, že tento předpoklad neplatí, byla hledána cesta, jak vyřešit problém s~určením hustoty. Na atomární hustotě a~koncentraci vodíku totiž přímo závisí síly přechodu pro fononové absorpce v~IR oblasti. 
Pokud bychom tedy například určili hustotu $N_\mathrm{a}$ příliš velkou, můžeme dojít k~příliš malé hodnotě vodíkové koncentrace a~naopak. Zkoumáním hustot získaných z~ERDA/RSB se ukázalo, že hmotnostní hustota se sice mezi vzorky poměrně liší, ale odpovídající atomové hustoty $N_\mathrm{a}$ jsou velmi podobné. 
Řešení celého problému bylo tedy takové, že atomová hustota $N_\mathrm{a}$ je sice ve fitu volný parametr, ale společný pro všechny vrstvy. Konkrétně byla atomová hustota $N_\mathrm{a}$ určena při kalibračním multisample fitu a~poté byla zafixována pro ostatní vzorky.

\section{Přechodová vrstva, substrát a~vrstva ze zadní strany}
Ve všech fitech byla mezi substrátem a~DLC vrstvou uvažována i~mezivrstva. Ta vzniká díky implantaci energetických iontů uhlíku a~vodíku do krystalického křemíku při růstu vrstvy. 
Při jejím modelování byl použit jednoduchý trojparametrický PJDOS model s~jedním gausovským píkem ve viditelné oblasti, který reprezentuje absorpci na defektních stavech. 
Parametry této přechodové vrstvy byly pro všechny vrstvy stejné a~byly určeny při kalibračním fitu. Jediný parametr, který se mezi jednotlivými vzorky lišil, byla tloušťka přechodové vrstvy $d_\mathrm{t}$. 
Parametry elektronové struktury přechodové vrstvy jsou v~tabulce \ref{tsb-params}. Index lomu přechodové vrstvy spolu s~indexem lomu substrátu a~indexem lomu jedné DLC vrstvy je na obrázku \ref{transition}. 
Je vidět, že index lomu je pro nízké energie někde mezi indexem lomu DLC vrstvy a~indexem lomu krystalického křemíku, což je přesně očekávaná hodnota. Pro vyšší energie jde potom index lomu k~nule, což je pravděpodobně způsobeno přílišným zjednodušením modelu přechodové vrstvy.

\begin{table}[htbp]
 \centering
	\renewcommand{\tabcolsep}{4pt}
 \begin{tabular}{lcccccccc}
\hline
parametry & \multicolumn{4}{c}{Kalibrační multifit} & \multicolumn{4}{c}{Samostatné fity}\\
 & CH30A & CH83A & CH88A & CH90A & CH30C & CH30D & CH87A & CH89A\\
\hline
\multicolumn{9}{l}{Přechodová vrstva}\\
\hline
$Q_\mathrm{t}$\,[eV$^{3/2}$]$^\mathrm{a}$ & 13.7 & 13.7 & 13.7 & 13.7 & 13.7 & 13.7 & 13.7 & 13.7\\
$E_\mathrm{g}$\,[eV]$^\mathrm{a}$ & 1.4 & 1.4 & 1.4 & 1.4 & 1.4 & 1.4 & 1.4 & 1.4\\
$E_\mathrm{h}$\,[eV]$^\mathrm{a}$ & 5.2 & 5.2 & 5.2 & 5.2 & 5.2 & 5.2 & 5.2 & 5.2\\
$Q_\mathrm{G}$\,[eV$^{3/2}$]$^\mathrm{a}$ & 2.8 & 2.8 & 2.8 & 2.8 & 2.8 & 2.8 & 2.8 & 2.8\\
$E_\mathrm{G}$\,[eV]$^\mathrm{a}$ & 1.8 & 1.8 & 1.8 & 1.8 & 1.8 & 1.8 & 1.8 & 1.8\\
$E_\mathrm{G}$\,[eV]$^\mathrm{a}$ & 0.5 & 0.5 & 0.5 & 0.5 & 0.5 & 0.5 & 0.5 & 0.5\\
$d_\mathrm{t}$\,[nm] & 3.1 & 40.9 & 23.0 & 3.2 & 3.5 & 4.7 & 39.3 & 4.9\\
\multicolumn{9}{l}{}\\
\multicolumn{9}{l}{Substrát c-Si}\\
\hline
$f_\mathrm{O}$\,[10$^{-6}$]$^\mathrm{b}$ & - & - & - & - & 22.80 & 22.58 & 20.77 & 21.53\\
$d_\mathrm{s}$\,[$\mu$m] & 327 & 318 & 314 & 328 & 353 & 370 & 381 & 372\\
\multicolumn{9}{l}{}\\
\multicolumn{9}{l}{Vrstva na zadní straně SiO$_2$}\\
\hline
$d_\mathrm{b}$\,[nm] & 1.4 & 1.2 & 2.0 & 4.3 & 1.3 & 1.5 & 0.8 & 0.3\\

\hline
\multicolumn{9}{l}{$^\mathrm{a}$ parametr byl volný pouze při kalibračním fitu a byl společný pro všechny vrstvy}\\
\multicolumn{9}{l}{$^\mathrm{b}$ pro kalibrační multifit byly použity tabulkové hodnoty}\\
\end{tabular}

 \caption{Parametry přechodové vrstvy, substrátu a~vrstvy SiO$_2$ ze zadní strany.}
\label{tsb-params}
\end{table}

\begin{figure}[htbp]
	\includegraphics[width=\linewidth]{grafy/transition.pdf}
	\caption{Porovnání indexu lomu DLC vrstvy (vzorek CH30A), přechodové vrstvy a~krystalického křemíku.} 
	\label{transition}
\end{figure}

Tloušťka přechodové vrstvy se mezi jednotlivými vzorky velmi liší, od přibližně 3\,nm pro vzorky CH30A, CH30C, CH30D a~CH90A až po 40\,nm pro vzorky CH83A a~CH87A. Přestože bychom očekávali hlavně závislost na depozičních podmínkách, konkrétně na předpětí a~tlaku na nichž přímo závisí energie dopadajícího iontu a~tím pádem i~hloubka implantace, žádná taková závislost není patrná. 
Můžeme si ale povšimnout, že větší tloušťka přechodových vrstev nastává hlavně pro tlusté DLC vrstvy. Možné vysvětlení tedy opět může být nehomogenita, artefakt vzniklý během fitování a~nebo to, že u tlustších vrstev trvala depozice déle a~docházelo k~výraznějšímu zahřívaní vzorku.

Co se týče modelování substrátu z~krystalického křemíku, plánem bylo pro všechny vzorky použít model PJDOS c-Si, hlavně kvůli fitování kyslíkových příměsí. 
Nevýhodou modelu je podobně jako u modelu nehomogenity jeho velká výpočetní náročnost a~proto při kalibračním fitu, kde bylo potřeba určit přes sto parametrů pro čtyři vrstvy najednou, trvala jedna iterace při použití PJDOS c-Si modelu (při souběžném počítání na 32 CPU) téměř patnáct minut (konkrétně pro kalibrační multisample fit byly potřeba stovky iterací a~ruční ladění parametrů, než došlo k~dobré shodě s~experimentem). 
Proto byly pro kalibrační multisample fit použity tabulkové hodnoty, ale pro další fity, kde bylo volných parametrů mnohem méně, už byl model 
PJDOS c-Si normálně používán. V tabulce \ref{fitparams} jsou jednak koncentrace intersticiálního kyslíku a~také tloušťka substrátu. Je vidět, že hlavně s~ta\-bul\-ko\-vými hodnotami vycházejí tloušťky poněkud méně, než je výrobcem udávaná hodnota (360\,$\mu$m).

Co se týče SiO$_2$ vrstvy ze zadní strany, všechny tloušťky jsou v~rozmezí 1--4\,nm, což odpovídá očekáváním.


\section{Popis vibračních stavů v~IR oblasti}
Hlavním cílem této práce je určení celkové koncentrace vodíku v~DLC vrstvách. Pro to je klíčová správná identifikace jednotlivých absorpčních píků a~správné určení jejich relativní síly přechodu.

Pro píky sp$^3$CH$_{\{1,2,3\}}$ skupin je jejich identifikace poměrně jednoduchá. Valenční vibrace sp$^3$CH jsou v~literatuře udávány buď na 2915\,cm$^{-1}$ \cite{Zajickova2011, Dischler1983} případně 2920\,cm$^{-1}$ \cite{Robertson2002, Ristein1998}. Skupina sp$^3$CH$_2$ má udávaný antisymetrický pík na 2920\,cm$^{-1}$ \cite{Dischler1983, Robertson2002, Ristein1998} nebo na 2925\,cm$^{-1}$ \cite{Zajickova2011}, symetrický pak na 2855\,cm$^{-1}$ \cite{Robertson2002, Ristein1998, Zajickova2011} nebo 2850\,cm$^{-1}$ \cite{Dischler1983}. Podobná shoda panuje i~u antisymetrických vibrací skupiny sp$^3$CH$_3$ 2960\,cm$^{-1}$ \cite{Zajickova2011, Dischler1983} případně 2955\,cm$^{-1}$ \cite{Robertson2002, Ristein1998}. U symetrických vibrací skupiny sp$^3$CH$_3$ se literatura začíná poněkud rozcházet a~udává buď 2885\,cm$^{-1}$ \cite{Robertson2002, Ristein1998} nebo 2870\,cm$^{-1}$ \cite{Zajickova2011, Dischler1983}. 

Horší situace panuje při identifikace vibračních píků skupin sp$^2$CH$_x$. Literatura udává olefinickou sp$^2$CH skupinu buď na 2990\,cm$^{-1}$ \cite{Ristein1998}, 2990--3000\,cm$^{-1}$ \cite{Robertson2002}, 3030\,cm$^{-1}$ \cite{Zajickova2011}, nebo 3045\,cm$^{-1}$ \cite{Dischler1983}. Nejhorší situace je potom při identifikaci vibračních píků skupiny sp$^2$CH$_2$. Antisymetrické vibrace této skupiny jsou udávány na 3000\,cm$^{-1}$ \cite{Zajickova2011}, 3020\,cm$^{-1}$ \cite{Dischler1983}, 3085\,cm$^{-1}$ \cite{Robertson2002}, nebo 3082\,cm$^{-1}$ \cite{Ristein1998}. Symetrické vibrace stejné skupiny pak literatura udává na 2950\,cm$^{-1}$ \cite{Dischler1983}, 2975\,cm$^{-1}$ \cite{Robertson2002, Ristein1998}, nebo 3080\,cm$^{-1}$ \cite{Zajickova2011},.

\begin{table}[!tb]
 \centering
 \begin{tabular}{lccccccccc}
\hline
parametry & & \multicolumn{4}{c}{Kalibrační multisample fit} & \multicolumn{4}{c}{Samostatné fity}\\
 &  & 30A & 83A & 88A & 90A & 30C & 30D & 87A & 89A\\
\hline
$\alpha_\mathrm{sp3CH3a}\,^\ast$ & $[10^{-3}]$ & 3.7 & 3.7 & 3.7 & 3.7 & 3.7 & 3.7 & 3.7 & 3.7\\
$\nu_\mathrm{sp3CH3a}^\dagger$ & $[\mathrm{cm}^{-1}]$ & 2955 & 2955 & 2955 & 2955 & 2955 & 2955 & 2955 & 2955\\
$\alpha_\mathrm{sp3CH3s} ^\ast$ & $[10^{-3}]$ & 3.19 & 3.19 & 3.19 & 3.19 & 3.19 & 3.19 & 3.19 & 3.19\\
$\nu_\mathrm{sp3CH3s} ^\dagger$ & $[\mathrm{cm}^{-1}]$ & 2870 & 2870 & 2870 & 2870 & 2870 & 2870 & 2870 & 2870\\
$\beta_\mathrm{sp3CH3}$ & $[10\,\mathrm{cm}^{-1}]$ & 6.62 & 5.5 & 5.21 & 8.1 & 4 & 5.6 & 5.5 & 5.1\\
$\alpha_\mathrm{sp3CH3bs} ^\ast$ & $[10^{-3}]$ & 1.05 & 1.05 & 1.05 & 1.05 & 1.05 & 1.05 & 1.05 & 1.05\\
$\nu_\mathrm{sp3CH3bs} ^\dagger$ & $[\mathrm{cm}^{-1}]$ & 1375 & 1375 & 1375 & 1375 & 1375 & 1375 & 1375 & 1375\\
$\beta_\mathrm{sp3CH3bs}$ & $[10\,\mathrm{cm}^{-1}]$ & 3.1 & 3.3 & 3.23 & 4.07 & 2.5 & 2.3 & 3 & 3.0\\
$\alpha_\mathrm{sp3CH2a} ^\ast$ & $[10^{-3}]$ & 3.5 & 3.5 & 3.5 & 3.5 & 3.5 & 3.5 & 3.5 & 3.5\\
$\nu_\mathrm{sp3CH2a} ^\dagger$ & $[\mathrm{cm}^{-1}]$ & 2920 & 2920 & 2920 & 2920 & 2920 & 2920 & 2920 & 2920\\
$\alpha_\mathrm{sp3CH2s} ^\ast$ & $[10^{-3}]$ & 3.11 & 3.11 & 3.11 & 3.11 & 3.11 & 3.11 & 3.11 & 3.11\\
$\nu_\mathrm{sp3CH2s} ^\dagger$ & $[\mathrm{cm}^{-1}]$ & 2850 & 2850 & 2850 & 2850 & 2850 & 2850 & 2850 & 2850\\
$\beta_\mathrm{sp3CH2}$ & $[10\,\mathrm{cm}^{-1}]$ & 5.8 & 10 & 9.13 & 7.0 & 4.9 & 3.9 & 10.8 & 9.4\\
$\alpha_\mathrm{sp3CH} ^\ast$ & $[10^{-3}]$ & 11.1 & 11.1 & 11.1 & 11.1 & 11.1 & 11.1 & 11.1 & 11.1\\
$\nu_\mathrm{sp3CH} ^\dagger$ & $[\mathrm{cm}^{-1}]$ & 2915 & 2915 & 2915 & 2915 & 2915 & 2915 & 2915 & 2915\\
$\beta_\mathrm{sp3CH}$ & $[10\,\mathrm{cm}^{-1}]$ & 11.03 & 4.62 & 5.02 & 9.5 & 11.0 & 11.3 & 4.8 & 5.0\\
$\alpha_\mathrm{sp2CH2a} ^\ast$ & $[10^{-3}]$ & 1,9 & 1,9 & 1,9 & 1,9 & 1,9 & 1,9 & 1,9 & 1,9\\
$\nu_\mathrm{sp2CH2a} ^\dagger$ & $[\mathrm{cm}^{-1}]$ & 3080 & 3080 & 3080 & 3080 & 3080 & 3080 & 3080 & 3080\\
$\alpha_\mathrm{sp2CH2s} ^\ast$ & $[10^{-3}]$ & 1.4 & 1.4 & 1.4 & 1.4 & 1.4 & 1.4 & 1.4 & 1.4\\
$\nu_\mathrm{sp2CH2s} ^\dagger$ & $[\mathrm{cm}^{-1}]$ & 2975 & 2975 & 2975 & 2975 & 2975 & 2975 & 2975 & 2975\\
$\beta_\mathrm{sp2CH2}$ & $[10\,\mathrm{cm}^{-1}]$ & 14 & 12.0 & 2.96 & 10 & 8 & 8 & 8.9 & 10\\
$\alpha_\mathrm{sp2CH} ^\ast$ & $[10^{-3}]$ & 8.79 & 8.79 & 8.79 & 8.79 & 8.79 & 8.79 & 8.79 & 8.79\\
$\nu_\mathrm{sp2CH} ^\dagger$ & $[\mathrm{cm}^{-1}]$ & 3000 & 3000 & 3000 & 3000 & 3000 & 3000 & 3000 & 3000\\
$\beta_\mathrm{sp2CH}$ & $[10\,\mathrm{cm}^{-1}]$ & 8.91 & 9.6 & 10.5 & 12.0 & 10.0 & 10.2 & 10.3 & 9.6\\

\hline
\multicolumn{10}{l}{$^\ast$ parametr byl volný pouze při kalibračním fitu a byl společný pro všechny vrstvy}\\
\multicolumn{10}{l}{$^\dagger$ zafixováno na hodnoty z literatury, viz kapitola \ref{sirky}}\\
\end{tabular}

 \caption{Parametry gausovských píků reprezentujících vibrační módy CH skupin, které byly zahrnuty do určování celkové koncentrace vodíku. Názvy vrstev byly kvůli velké šířce tabulky zkráceny.}
\label{H-peaks}
\end{table}

Vzhledem k~velkému překryvu píků a~jejich značné šířce není možné zvolit polohy píků jako volný parametr, proto bylo potřeba polohy píků zafixovat. Přitom byl brán ohled na výše uvedený průzkum literatury a~také na shodu fitu pro danou polohu absorpčního píku s~měřenými daty. 
Konečné polohy píků používané při fitování byly nakonec 2915\,cm$^{-1}$ pro valenční vibrace skupiny sp$^3$CH, 2850\,cm$^{-1}$ pro symetrické a~2920\,cm$^{-1}$ pro antisymetrické valenční vibrace skupiny sp$^3$CH$_2$, 2870\,cm$^{-1}$ pro symetrické a~2955\,cm$^{-1}$ pro antisymetrické valenční vibrace skupiny sp$^3$CH$_3$. 3000\,cm$^{-1}$ pro valenční vibrace skupiny sp$^2$CH a~nakonec 2975\,cm$^{-1}$ pro symetrické vibrace a~3080\,cm$^{-1}$ pro antisymetrické vibrace skupiny sp$^2$CH$_2$. Přehled hodnot je v~tabulce \ref{H-peaks}.

\begin{figure}[!tb]
	\includegraphics[width=\linewidth]{grafy/Si-detail.pdf}
	\caption{Detail propustnosti vrstvy CH30A v~oblasti 1200--1700\,cm$^{-1}$ spolu s~ima\-gi\-nární částí dielektrické funkce krystalického křemíku ve stejné oblasti.}
	\label{Si-detail}
\end{figure}

Kromě valečních vibrací v~oblasti 2800--3100\,cm$^{-1}$ mají CH skupiny i~značné množství deformačních vibrací v~oblasti 1375--1480\,cm$^{-1}$. Teoreticky by tedy bylo možné pro určování celkové koncentrace vodíku kromě valenčních vibrací používat i~deformační. Valenční vibrace mají nicméně tu výhodu, že v~oblasti 2800--3100\,cm$^{-1}$ použitý substrát neabsorbuje a~veškerá absorpce tedy přísluší pouze CH skupinám. 
V oblasti deformačních vibrací to bohužel není pravda, krystalický křemík zde již absorbuje a~je velmi těžké odlišit absorpci substrátu od absorpce vrstvy. Toto ilustruje obrázek \ref{Si-detail}, na kterém nejsou víceméně jiné píky než ty způsobené křemíkovou absorpcí vůbec okem rozeznatelné. Z~tohoto důvodu byl do určování celkové koncentrace vodíku zahrnut pouze pík symetrických deformačních vibrací skupiny sp$^3$CH$_3$ na 1375\,cm$^{-1}$ \cite{Robertson2002}. Ten má oproti ostatním píkům v~této oblasti alespoň tu výhodu, že jako jediný je samostatný a~nepřekrývá se s~ostatními vibračními módy.

\begin{table}[!tbp]
 \centering
	\renewcommand{\tabcolsep}{4pt}
 \begin{tabular}{lccccccccc}
\hline
parametry & & \multicolumn{4}{c}{Kalibrační multisample fit} & \multicolumn{4}{c}{Samostatné fity}\\
 &  & 30A & 83A & 88A & 90A & 30C & 30D & 87A & 89A\\
\hline
$N_\mathrm{sp2C,A}$ & $[10^{-6}\,\mathrm{eV}^2]$ & 301 & 82 & 128 & 373 & 43 & 270 & 106 & 236\\
$\nu_\mathrm{sp2C,A}$ & $[\mathrm{cm}^{-1}]$ & 1300 & 1303.3 & 1308 & 1300 & 1301 & 1284 & 1305 & 1305\\
$\beta_\mathrm{sp2C,A}$ & $[10^1\mathrm{cm}^{-1}]$ & 20 & 10 & 14 & 36 & 12.8 & 18 & 12.2 & 19\\
$N_\mathrm{sp2C,B}$ & $[10^{-6}\,\mathrm{eV}^2]$ & 76.3 & 76 & 117 & 158 & 61 & 138 & 72 & 153\\
$\nu_\mathrm{sp2C,B}$ & $[\mathrm{cm}^{-1}]$ & 1602.8 & 1603 & 1600 & 1627.3 & 1596 & 1605 & 1600 & 1597\\
$\beta_\mathrm{sp2C,B}$ & $[10^1\mathrm{cm}^{-1}]$ & 8.9 & 12.3 & 17 & 15 & 7.6 & 10 & 10 & 18\\
$N_\mathrm{sp2C,C}$ & $[10^{-6}\,\mathrm{eV}^2]$ & 55 & 31 & 80 & 27 & 71 & 46 & 56 & 101\\
$\nu_\mathrm{sp2C,C}$ & $[\mathrm{cm}^{-1}]$ & 1682 & 1680 & 1679 & 1681 & 1681 & 1708 & 1680.2 & 1680.5\\
$\beta_\mathrm{sp2C,C}$ & $[10^1\mathrm{cm}^{-1}]$ & 10 & 11 & 13.4 & 12 & 13 & 5.9 & 11 & 14\\
$N_\mathrm{sp3CHxy}$ & $[10^{-6}\,\mathrm{eV}^2]$ & 80 & 102 & 103 & 53 & 126 & 154 & 112 & 86\\
$\nu_\mathrm{sp3CHxy}$ & $[\mathrm{cm}^{-1}]$ & 1448 & 1445 & 1444.7 & 1442 & 1450 & 1444 & 1446 & 1446\\
$\beta_\mathrm{sp3CHxy}$ & $[10^1\mathrm{cm}^{-1}]$ & 4.3 & 5.2 & 5.3 & 4.5 & 5 & 6 & 5.5 & 5\\
\hline
\end{tabular}

 \caption{Parametry gausovských píků reprezentujících vibrační módy, které nebyly zahrnuty do určování celkové koncentrace vodíku. Názvy vrstev byly kvůli velké šířce tabulky zkráceny.}
\label{H-peaks2}
\end{table}

Kromě výše zmíněných se v~IR spektru vyskytují i~další píky, které buď nejsou pro určování vodíkové koncentrace důležité, nebo jsme se je, kvůli výše uvedeným důvodům, rozhodli do určování vodíkové koncentrace nezahrnout. Přesto jsou stále důležité pro celkovou dobrou shodu fitu s~měřenými daty a~z toho plynoucí dobrou shodu v~oblasti valenčních vibrací, které nás zajímají nejvíce. Síly přechodu těchto píků jíž nejsou, na rozdíl od předchozích, vázány na koncentraci příslušejících skupin a~jsou volný parametr. 
Nemá proto cenu vyjadřovat je jako relativní parciální síly přechodu $\alpha_p$, ale místo toho jako integrální parciální síly přechodu pro jednotlivé píky $N_p$. Jedná se hlavně o vibrace dvojných vazeb C--C sp$^2$ uhlíku na 1300\,cm$^{-1}$, 1600\,cm$^{-1}$ a~1680\,cm$^{-1}$ \cite{Robertson2002, Theye2001, Zajickova2011} (v~tabulce \ref{H-peaks2} jsou značeny jako sp$^2$C$_{\{A,B,C\}}$) a~další pík okolo 1450\,cm$^{-1}$ (značený jako sp$^3$CHxy), který reprezentuje zbývající vibrace CH skupin v~této oblasti.

Důležitým výsledkem této práce jsou relativní parciální síly přechodu $\alpha_p$, kde $p$ značí jednotlivé sp$^x$CH$_y$ skupiny. Ty jsou také shrnuty v~tabulce \ref{H-peaks}. Jak již bylo uvedeno dříve, nejsem si vědom žádné práce, která by systematicky zkoumala síly přechodu, síly oscilátoru, případně efektivní náboje uhlovodíkových skupin v~DLC vrstvách, nelze tedy jednoduše porovnat získané výsledky s~literaturou. 
Získané výsledky se nicméně alespoň pokusím porovnat s~efektivními náboji polymerních uhlovodíkových vrstev, jejichž přehled je v~\cite{Heitz1998}. Pro přepočet mezi relativními silami přechodu a~efektivními náboji lze použít vzorec (\ref{efch2str}) a~dále je nutné uvážit, že zatímco efektivní náboj se vztahuje k~danému vibračnímu módu sp$^x$CH$_y$ skupiny, relativní síla přechodu se vztahuje ke koncentraci vodíku v~dané konfiguraci a~danému vibračnímu módu. 
Tedy například stejné hodnoty $f_\mathrm{sp^3CH}$ a~$f_\mathrm{sp^3CH2}$ znamenají, že vodík je rozložený mezi oběma skupinami rovnoměrně, jelikož ale sp$^3$CH$_2$ skupina obsahuje dva vodíky a~sp$^3$CH skupina jen jeden, bude počet sp$^3$CH$_2$ skupin dvakrát menší. Při přepočtu je tedy potřeba vydělit efektivní náboje sp$^x$CH$_2$ skupin dvěma a~sp$^x$CH$_3$ skupin třemi. 

Z určených relativních sil přechodu je nejsilnější $\alpha_\mathrm{sp^3CH}$, to se shoduje s~\cite{Francis1950, Wexler1967}, ale neodpovídá výsledkům z~polypropylenu \cite{Heitz1998}, kde vycházejí relativních síly přechodu velmi podobné pro všechny sp$^3$CH$y$ skupiny. Výsledkům z~polypropylenu naopak odpovídá poměr mezi symetrickými a~antisymetrickými vibracemi, ty jsou jen přibližně o 20\,\% silnější což odpovídá zjištěným poměrům určeným v~této práci. Francis \cite{Francis1950} a~Wexler \cite{Wexler1967} pak udávají antisymetrické vibrace přibližně o polovinu silnější než symetrické. Jediné uvažované symetrické deformační vibrace skupiny sp$^3$CH$_3$ jsou udávány asi třikrát slabší než valenční, to také bobře odpovídá výsledkům v~této práce.  
Pro vodík vázaný na sp$^2$ uhlících je situace ještě poněkud složitější, protože kromě stavu primárního uhlíku zde závisí relativní síla přechodu významně i~na sekundárních uhlících \cite{Heitz1998}. Literatura nicméně udává, že valenční vibrace sp$^2$CH skupiny jsou podobně silné jako u sp$^3$CH a~naopak sp$^2$CH$_2$ skupina je udávána jako velmi slabá, což opět souhlasí s~výsledky z~tabulky \ref{H-peaks}. Je ale opět potřeba zdůraznit, že výše citované práce se zabývají polymerními uhlovodíkovými vrstvami a~ne tvrdými amorfními DLC.   

\label{sirky}
Komplikovaná je i~analýza šířky píků. Dischler \cite{Dischler1983} udává šířky píků valenčních sp$^x$CH$_y$ vibrací přibližně od 70\,cm$^{-1}$ do 90\,cm$^{-1}$, jinde jsou píky udávány užší (44\,cm$^{-1}$ pro sp$^2$CH$_{\{1,2\}}$ a~60\,cm$^{-1}$ pro sp$^3$CH$_{\{1,2,3\}}$ skupiny \cite{Zajickova2011}). Šířka píků je nicméně velmi závislá na struktuře materiálu, především na homogenitě a~na pnutí ve vrstvě. 
To vede k~rozšiřování píků u tvrdších vrstev \cite{Ristein1998}. Pološířky určené z~fitů jsou někdy mírně větší, ale většinou odpovídají výše uvedeným hodnotám. Pološířka píku reprezentujícího valenční vibrace sp$^3$CH$_3$ skupiny je podobná pro většinu vrstev okolo 50--60\,cm$^{-1}$, což se dobře shoduje s~výše uvedenými výsledky.
Podobná shoda je i~pro pološířku deformačních vibrací stejné skupiny. Trochu horší je situace u vibrací skupin sp$^3$CH$_2$ a~sp$^3$CH, kde se pološířka mění od vrstvy k~vrstvě. Pro sp$^2$CH skupiny je pak pološířka pro všechny vrstvy velmi podobná, přibližně 100\,cm$^{-1}$. Pro sp$^2$CH$_x$ vibrace se pološířky mění velmi divoce, nicméně je potřeba si všimnout, že podle tabulky \ref{fitparams} obsahují nezanedbatelné množství CH$_2$ skupin jen tři vrstvy a~to CH83A, CH87A, a~CH89A. Pro ostatní vrstvy tedy pološířky této skupiny prakticky nemají význam.

\section{Celková koncentrace vodíku}
Přejděme nyní konečně k~hlavnímu bodu této práce a~to k~analýze celkového množství vodíku ve vrstvách. Celkové koncentrace vodíku a~koncentrace vodíku v~jednotlivých skupinách jsou uvedeny nahoře v~tabulce \ref{fitparams}. 
Jsou tam také uvedeny hodnoty z~ERDA. Pro první čtyři vrstvy jsou tyto hodnoty stejné, na těchto vrstvách byla provedena kalibrace relativních sil přechodu pro jednotlivé vibrační módy odpovídajících skupin. 
Pro zbylé čtyři vrstvy byly použity získané hodnoty a~celková vodíková koncentrace byla volným parametrem při fitování. Rozklad celkové dielektrické funkce na jednotlivé příspěvky v~dané oblasti je pro čtyři vrstvy použité při kalibraci na obrázku \ref{deconvolute} a~na obrázku \ref{deconvolute2} pro ostatní vrstvy.
 
\begin{figure}[htbp]
	\includegraphics[width=\linewidth]{grafy/T-multifit.pdf}
	\caption{IR propustnost vrstev CH30A, CH83A, CH88A a~CH90A.}
	\label{T-multifit}
\end{figure}

\begin{figure}[htbp]
	\includegraphics[width=\linewidth]{grafy/T-multifit2.pdf}
	\caption{IR propustnost vrstev CH30C, CH30D, CH87A a~CH89A.}
	\label{T-multifit2}
\end{figure}

\begin{figure}[htbp]
	\includegraphics[width=\linewidth]{grafy/T-detail-multifit.pdf}
	\caption{Detail IR propustnosti vrstev CH30A, CH83A, CH88A a~CH90A v~oblasti valenčních vibrací CH skupin.}
	\label{T-detail}
\end{figure}

\begin{figure}[htbp]
	\includegraphics[width=\linewidth]{grafy/T-detail-multifit2.pdf}
	\caption{Detail IR propustnosti vrstev CH30C, CH30D, CH87A a~CH89A v~oblasti valenčních vibrací CH skupin.}
	\label{T-detail2}
\end{figure}

\begin{figure}[htbp]
	\includegraphics[width=\linewidth]{grafy/deconv-multifit.pdf}
	\caption{Příspěvky jednotlivých valenčních vibračních módů k dielektrické funkci v~oblasti valenčních vibrací CH skupin pro vrstvy CH30A, CH83A, CH88A a~CH90A.}
	\label{deconvolute}
\end{figure}

\begin{figure}[htbp]
	\includegraphics[width=\linewidth]{grafy/deconv-multifit2.pdf}
	\caption{Příspěvky jednotlivých valenčních vibračních módů k dielektrické funkci v~oblasti valenčních vibrací CH skupin pro vrstvy CH30C, CH30D, CH87A a~CH89A.}
	\label{deconvolute2}
\end{figure}

Shody fitu s~naměřenými daty jsou pro celé IR spektrum na obrázcích \ref{T-detail} a~\ref{T-detail2}. Na obrázcích \ref{T-detail} a~\ref{T-detail2} pak můžeme vidět detail v~oblasti valenčních vibrací CH skupin. Je vidět, že fity ve zkoumané oblasti velmi dobře sedí, což je základní podmínka pro kvalitní výsledky. Pouze u vrstvy CH30D je vidět nesoulad pro vlnočty nad 4000\,cm$^{-1}$, to je pravděpodobně způsobeno tím, že vzorek není dokonale planparalelní. Pro kompenzaci a~dobrou shodu v~oblasti valenčních vibrací byl do fitu této vrstvy přidán široký gausovský pík, který částečně kompenzoval nesoulad a~zajistil dobrou shodu v~oblasti 2800--3100\,cm$^{-1}$. Dále je potřeba podotknout, že především pro tenké vrstvy, kde máme málo signálu, už v~infračervených spektrech začíná být patrný šum.  
Porovnáním hodnot z~RBS/ERDA s~nafitovanými hodnotami se ukazuje, že pro celkovou vodíkovou koncentraci se podařilo dosáhnout velmi dobré shody.

Nejlepší shoda je pro vzorek CH30C, kde se hodnota 30,6\,\% z~EDRA liší od 31,5\,\% z~optického měření jen minimálně. Podobná dobrá shoda je u vzorku CH30D, 30,3\,\% z~EDRA oproti 31,5\,\% z~optiky. Tyto výsledky nicméně nejsou překvapující, protože tyto vzorky jsou v~podstatě identické se vzorkem CH30A, na kterém byla prováděna kalibrace. Horší situace je pak u vzorků s~tlustšími vrstvami. Tam je rozdíl mezi 48\,\% z~ERDA a~44,5\,\% z~optiky u CH87A a~35,0\,\% z~ERDA oproti 39,9\,\% určených z~optického měření u CH89A již poněkud větší, nicméně stále se jedná o velmi dobrou shodu.

Z tabulky je jasně patrné, že pro malé celkové vodíkové koncentrace převažuje vodík v~sp$^3$CH skupině. Vrstvy CH30A, CH30C a~CH30D, které mají všechny koncentraci vodíku okolo 30\,\% mají z~toho postupně 13,4\,\%, 17,0\,\% a~17,6\,\% právě v~sp$^3$CH. Tyto vrstvy byly deponovány při největším výkonu z~čehož plyne vyšší hustota a~tvrdost vrstvy a~tím pádem je tento výsledek očekávaný. 
Můžeme tedy říci, že pro tvrdé vrstvy převažuje právě vodík sp$^3$CH. Poněkud jiná je situace u vrstvy CH90A. Ta má sice 31\,\% celkového vodíku, ale v~sp$^3$CH skupině pouze 6\,\%. To je možno vysvětlit tak, že vrstva byla deponována při nižším výkonu a~očekávali bychom nižší hustotu a~tvrdost a~také větší množství vodíku v~koncových sp$^3$CH$_3$ skupinách, které zmenšují celkovou provázanost uhlíkové sítě a~tím samozřejmě i~tvrdost. Jak už ale bylo uvedeno u diskuse výsledků z~RBS/ERDA, konkrétně vrstva CH90A se pro iontové metody ukázala být velmi problematická, a~tak je možné, že byla celková koncentrace určena příliš nízká. Tomu by odpovídalo i~to, že rozložení vodíku mezi jednotlivé skupiny určené z~optických měření je velmi podobné vrstvám, které mají celkovou koncentraci vodíku mnohem vyšší.

U všech ostatních vrstev je patrný trend, že s~přechodem k~celkové větší koncentraci vodíku obecně dochází ke shlukování vodíku, tedy k~růstu množství vodíku v~sp$^3$CH$_2$, sp$^3$CH$_3$ a~sp$^2$CH$_2$ skupinách. Vodík v~sp$^2$CH$_2$ skupinách můžeme pozorovat pouze u vrstev s~větším celkovým množstvím vodíku, hlavně u CH83A a~CH87A. Tyto vrstvy také mají největší množství vodíku v~sp$^3$CH$_3$ skupinách, 12,3\,\% a~11,3\,\%. Nejméně vodíku v~sp$^3$CH$_3$ skupinách mají vrstvy CH30A -- 3,6\,\%, CH30C -- 3,5\,\% a~CH30D 5,1\,\%. Koncentrace vodíku v~sp$^2$CH skupinách je pak na celkové koncentraci vodíku ve vrstvě víceméně nezávislá. Přehledně to ukazuje obrázek \ref{Hdif}.

Není zcela jasné, proč mají vrstvy CH83A a~CH87A o tolik větší celkovou koncentraci vodíku než vrstvy CH88A, CH89A a~CH90A. Depoziční podmínky jsou pro všechny tyto vrstvy téměř identické (s vyjímkou vyššího tlaku a~přimícháním argonu místo vodíku do metanu pro depozici vrstvy CH83A). 
Možné vysvětlení je právě nespolehlivost reaktoru, která se projevuje měnícím se předpětím a~byla již diskutována v~experimentální kapitole, nicméně pro určení závislosti parametrů na depozičních podmínkách bychom potřebovali více vzorků a~není to náplní této práce.

Pro žíhané vrstvy CH30C a~CH30D není pozorován žádný významný úbytek vodíku a~oproti očekávání dochází k~nárůstu koncentrace vodíku v~sp$^3$CH skupině a~především k~úbytku vodíku v~sp$^2$CH skupině, ačkoli při zahřívání bychom očekávali spíše grafitizaci vrstvy. Možné vysvětlení je to, že právě vodík v~sp$^2$CH má nejhorší teplotní stabilitu, nicméně všechny změny jsou velmi malé a~pravděpodobně jsou na samé hranici citlivosti metody.

\begin{figure}[tbhp]
	\includegraphics[width=\linewidth]{grafy/Hdif.pdf}
	\caption{Rozdělení vodíku mezi jednotlivé skupiny v~závislosti na celkové koncentraci vodíku, červeně je znázorněn vzorek CH90A, který se od ostatních výrazně liší.}
	\label{Hdif}
\end{figure}

Co se týče srovnání s~jinými pracemi, nepodařilo se nalézt žádnou práci, která by určovala procentuální rozložení vodíku mezi jednotlivé skupiny z~optických měření. Donnet et al. \cite{Donnet1999} nicméně zkoumal DLC vrstvy pomocí nukleární magnetické resonance a~ukázal, že většina vodíku je navázána v~sp$^3$CH a~sp$^3$CH$_2$ skupinách. 
Toto se dobře shoduje s~výše uvedenými výsledky. Na rozdíl od této práce, nukleární magnetická resonance nenaměřila žádné sp$^3$CH$_3$ skupiny. Zkoumané vrstvy byly ale deponovány při výrazně vyšších předpětích (500\,V a~800\,V) a~byly použity jiné depoziční plyny (acetylen a~cyklohexan). Absence sp$^3$CH$_3$ skupin se pak dá vysvětlit očekávávanou vyšší tvrdostí pro použité předpětí. Zajímavá situace nastává u sp$^2$CH$_x$ skupin, protože NMR sice nedetekovala žádné sp$^2$CH skupiny, ale detekovala nezanedbatelné množství sp$^2$CH$_2$ skupin. 
Tento fakt se jednak neshoduje se zjištěními z~této práce a~také to neodpovídá modelu růstu vrstvy, podle něhož dochází při dopadu molekulárního iontu na rostoucí vrstvu k~jeho rozpadu na jednotlivé atomy, z~nichž každý se pak deponuje zvlášť. Při tomto postupu bychom totiž očekávali, že sp$^2$CH$_2$ skupiny budou důsledkem navázání vodíku na již existující sp$^2$CH skupinu. Druhou možností zůstává, že sp$^2$CH$_2$ skupiny vznikají nějakým jiným procesem.

\cleardoublepage
