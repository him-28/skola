\chapter{Úvod}
\setcounter{page}{1}
\pagenumbering{arabic}

Tato práce se zabývá optickou charakterizací tvrdých amorfních uhlovodíkových vrstev (DLC\footnote{Diamond-like-carbon}). Především pak se zaměřením na infračervenou transmisní spektroskopii a určování celkové vodíkové koncentrace ve vrstvě.

DLC vrstvy jsou hojně využívány především jako ochranné vrstvy pro svoji vysokou mechanickou tvrdost, chemickou inertnost, malé tření a optickou průhlednost. Typické použití je například jako ochranné vrstvy na skla, zrcadla a jiné optické komponenty, magnetické disků a písty do motorů \cite{Robertson2002}. Vodíková koncentrace je pak velmi důležitý parametr vrstvy, protože vodík výrazně snižuje pnutí ve vrstvě a zabraňuje delaminaci vrstvy. Příliš velká vodíková koncentrace pak naopak vede k malé tvrdosti vrstvy. 

Běžné postupy pro určování koncentrace vodíku v DLC vrstvách jsou například ERDA (Elastic Recoil Detection Analysis), jaderná magnetická rezonance (NMR), NRA (Nuclear Reaction Analysis) (FIXME: přeložit tohle všechno???), nebo infračervená (IR) spektroskopie \cite{Robertson2002}. Výhodou IR spektroskopie je především dobrá dostupnost, snadné použití pro tenké vrstvy a také fakt, že je jedná o nedestruktivní metodu. 

Nevýhodou optické charakterizace při určování vodíkové koncentrace je především složitá diagnostika. Především v oblasti valenčních vodíkových vibrací (2800--3100\,cm$^-1$) se v infračerveném spektru nachází velké množství poměrně širokých absorpčních píků, které je velmi těžké od sebe odlišit. Běžný způsob pro určování množství vodíku ve vrstvě je proto takový, že se po odečtení vlivu substrátu a interferencí získaný absorpční koeficient integruje přes oblast vodíkové absorpce a vynásobí se empiricky zjištěnou konstantou $A$
\begin{equation}
N = A \int \frac{\alpha(\omega)}{\omega} \mathrm{d}\omega \text{,}
\end{equation}
kde N je počet vodíkových atomů na jednotkový objem a $\omega$ je frekvence. Bohužel se ukázalo, že $A$ je silně závislá na typu materiálu a může se pohybovat od $2 \times 10^{20}$ cm\,$^{-2}$ pro měkké polymerní vrstvy do $8,5 \times 10^{20}$ cm\,$^{-2}$ pro nejtvrdší vrstvy \cite{jacob1996}. Toto je způsobeno tím, že se ve vrstvách liší rozložení vodíku mezi jednotlivými C--H skupinami, které mají rozdílnou absorpci. Pro správné určení vodíkové koncentrace je proto potřeba určit koncentraci jednotlivých uhlovodíkových skupin a z nich pak teprve můžeme určit celkový vodíkový obsah. 

K optické charakterizace DLC vrstev byly používány programy z balíčku newAD, vyvinuté Mgr. Danem Frantou PhD. a Mgr. Davidem Nečasem PhD. z ústavu fyzikální elektroniky. Konkrétně byl použit model DLC PJDOS, který výborně popisuje viditelnou a ultrafialovou oblast a umožňuje určení jednak klasických optických veličin jako index lomu a tloušťka vrstvy, ale také parametrů elektronové struktury a některé makroskopické parametry jako hustota, poměr sp2 a sp3 hybridizovaného uhlíku a vodíkovou koncentraci. Bohužel právě koncentrace vodíku je ve stávající podobě určována nevyhovujícím způsobem. Cílem této práce proto je upravit model PJDOS DLC tak, aby dával správné výsledky co se týče vodíkového obsahu, a aby ho bylo možno použít pro rutinní charakterizace vodíkových koncentrací DLC vrstev
 
\cleardoublepage

