\chapter*{Závěr}
\addcontentsline{toc}{chapter}{Závěr}

Během této diplomové práce byly analyzovány diamantu podobné uhlíkové vrstvy na křemíkovém substrátu připravené ve vysokofrekvenčním nízkotlakém kapacitně vázaném doutnavém výboji ze směsi metanu a vodíku. K analýze vrstev byla použita optická měření, jednak propustnost v infračervené oblasti a dále také odrazivost a elipsometrie ve viditelné a blízké ultrafialové oblasti. K analýze naměřených dat byl použit program newAD, model PJDOS DLC, který byl v rámci této práce rozšířen o detailní model absorpce uhlovodíkových skupin. 

Ve spolupráci v Ústavem jaderné fyziky AV ČR v Řeži byly vrstvy také zkoumány metodami RBS a ERDA. Pomocí hodnot koncentrace vodíku určených iontovými metodami byly kalibračním multisample fitem čtyř vrstev určeny relativní síly přechodu pro valenční vibrace uhlovodíkových skupin, ty byly dále použity jednak pro analýzu celkové vodíkové koncentrace zbývajících vrstev a také pro analýzu rozložení vodíku mezi jednotlivé sp$^x$CH$_y$ skupiny. 

Výsledky celkové vodíkové koncentrace získané pomocí optických měření se dobře shodují s výsledky získanými z RBS/ERDA. Pro vrstvy, které byly velmi podobné kalibračním vrstvám se výsledky lišily jen minimálně, ale i pro vrstvy které byly více odlišné se podařilo dosáhnout velmi dobré shody s RBS/ERDA.

I rozložení vodíku mezi jednotlivé skupiny odpovídá očekáváním. Ukázalo se, že většina vodíku je navázána na sp3 uhlík, což odpovídá závěrům z jiných prací, že vodík je potřeba ke stabilizaci sp$^3$ fáze. Dále jsou z výsledků pro sedm z osmi vrstev jasně patrné změny v rozložení vodíku mezi jednotlivé skupiny v závislosti na jeho celkové koncentraci. Při nízkých koncentracích je vodík převážně přítomen v sp$^3$CH a sp$^2$CH skupinách, sp$^3$CH$_2$ skupiny jsou přítomny v menším množství a vodíku vázaného v sp$^3$CH$_3$ skupinách je minimum. S rostoucí celkovou koncentraci dochází k nárůstu množství sp$^3$CH$_2$ a sp$^3$CH$_3$ skupin. Všechny tyto výsledky velmi dobře odpovídají teoretickým očekáváním, pro nejtvrdší vrstvy s málo vodíkem je vodík rozmístěn rovnoměrně, naopak s rostoucí vodíkovou koncentraci roste počet koncových skupin, které přispívají k menšímu propojení uhlíkové mřížky a snižování hustoty a tvrdosti vrstev.

Pro žíhané vrstvy dochází při zahřívání ke změně poměru $f_\mathrm{Csp3}/f_\mathrm{Csp2}$ a je patrná grafitizace vrstev. Žíháním vrstev až na 320\,$^\circ$ nedošlo z žádným výrazným změnám vodíkové koncentrace, pouze k velmi malým fluktuacím celkové vodíkové koncentrace a mírnému přeskupení vodíku v rámci jednotlivých skupin.

Optická charakterizace je velmi dobrý nástroj k charakterizaci tenkých vrstev a ukázalo se, že optická charakterizace může sloužit i jako vhodný nástroj k analýze vodíkové koncentrace. Případně jako kontrola vodíkové koncentrace získané jiným způsobem (RBS/ERDA). Nevýhodou je nicméně potřeba komplikovaných modelů dielektrické funkce a 

Optické charakterizaci tenkých vrstev bych se chtěl i nadále věnovat při svém doktorském studiu, především se zaměřením na studium struktury pevných látek a jeho vlivu na optické charakteristiky. Z dalších věcí, které by si zasloužily podrobnější zkoumání bych zmínil 
$\mu$ $\mathrm{\mu}$

\cleardoublepage
