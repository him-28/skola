\chapter*{Závěr}
\addcontentsline{toc}{chapter}{Závěr}

Během této diplomové práce byly analyzovány diamantu podobné uhlovodíkové vrstvy na křemíkovém substrátu připravené ve vysokofrekvenčním nízkotlakém kapacitně vázaném doutnavém výboji ze směsi metanu a~vodíku. K charakterizaci vrstev byla použita optická měření, jednak propustnost v~infračervené oblasti a~dále také odrazivost a~elipsometrie ve viditelné a~blízké ultrafialové oblasti. K analýze naměřených dat byl použit program newAD, model PJDOS DLC, který byl v~rámci této práce rozšířen o detailní model absorpce uhlovodíkových skupin. 

Ve spolupráci v~Ústavem jaderné fyziky AV ČR v~Řeži byly vrstvy také zkoumány metodami RBS a~ERDA. Optická data čtyř vybraných vrstev byla fitována multisample metodou a~byly určeny relativní síly přechodu pro valenční vibrace jednotlivých uhlovodíkových skupin, s využitím vodíkových koncentrací měřených metodou ERDA pro kalibraci. Získané hodnoty byly poté úspěšně použity jednak pro analýzu celkové vodíkové koncentrace dalších čtyř vrstev a~také pro analýzu rozložení vodíku mezi jednotlivé uhlovodíkové skupiny (sp$^3$CH, sp$^3$CH$_2$, sp$^3$CH$_3$, sp$^2$CH a~sp$^3$CH$_2$). 

Výsledky celkové vodíkové koncentrace získané pomocí optických měření se dobře shodují s výsledky získanými z RBS/ERDA. Pro vrstvy, které byly velmi podobné kalibračním vrstvám se výsledky lišily jen minimálně, ale i pro vrstvy, které byly více odlišné, se podařilo dosáhnout velmi dobré shody s RBS/ERDA.

Rozložení vodíku mezi jednotlivé skupiny odpovídá očekáváním. Ukázalo se, že většina vodíku je navázána na sp$^3$ uhlík. Toto odpovídá závěrům z jiných prací, které ukázaly, že vodík je potřeba ke stabilizaci sp$^3$ fáze. Dále jsou z výsledků pro sedm z osmi vrstev jasně patrné změny v~rozložení vodíku mezi jednotlivé skupiny v~závislosti na jeho celkové koncentraci. 
Při nízkých celkových koncentracích je vodík převážně přítomen v~sp$^3$CH a~sp$^2$CH skupinách. sp$^3$CH$_2$ skupiny jsou přítomny v~menším množství a~vodíku vázaného v~sp$^3$CH$_3$ skupinách je minimum. S rostoucí celkovou koncentrací dochází k~nárůstu množství sp$^3$CH$_2$ a~sp$^3$CH$_3$. 
Všechny tyto výsledky velmi dobře odpovídají teoretickým očekáváním. Pro nej\-tvrdší vrstvy s malou vodíkovou koncentrací je vodík rozmístěn rovnoměrně v~sp$^3$CH skupinách, naopak s rostoucí vodíkovou koncentrací roste počet koncových skupin, které přispívají k~menšímu propojení uhlíkové mřížky a~snižování hustoty a~tvrdosti vrstev.

Pro žíhané vrstvy dochází při zahřívání ke změně poměru $f_\mathrm{Csp3}/f_\mathrm{Csp2}$ a~je patrná grafitizace vrstev. Žíháním vrstev až na 320\,$^\circ$C nedošlo k~žádným výrazným změnám vodíkové koncentrace, pouze k~velmi malým fluktuacím celkové vodíkové koncentrace a~mírnému přeskupení vodíku v~rámci jednotlivých skupin.

Optická charakterizace je velmi dobrý nástroj k~charakterizaci tenkých vrstev a~ukázalo se, že může sloužit i jako vhodný nástroj k~analýze vodíkové koncentrace v~DLC vrstvách, případně jako kontrola vodíkové koncentrace získané jiným způsobem (RBS/ERDA). Nevýhodou je nicméně potřeba komplikovaných modelů dielektrické funkce a~značná náročnost analýzy naměřených dat.

Optické charakterizaci tenkých vrstev bych se chtěl i nadále věnovat při svém doktorském studiu, především se zaměřením na studium struktury pevných látek a~jeho vlivu na optické charakteristiky. Z dalších věcí, které by bylo zajímavé podrobněji zkoumat, ale do rozsahu této práce se nevešly, bych zmínil studium nehomogenity vrstev, podrobnější rozbor přechodové vrstvy, případně analýzu v~širším spektrálním rozsahu.

\cleardoublepage
