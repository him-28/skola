\chapter*{Závěr}
\addcontentsline{toc}{chapter}{Závěr}

Během této diplomové práce byly analyzovány diamantu podobné uhlíkové vrstvy na křemíkovém substrátu připravené ve vysokofrekvenčním nízkotlakém kapacitně vázaném doutnavém výboji ze směsi metanu a vodíku. K analýze vrstev byla použita optická měření, jednak propustnost v infračervené oblasti a dále také odrazivost a elipsometrie ve viditelné a blízké ultrafialové oblasti.

K analýze naměřených dat byl použit program newAD, model PJDOS DLC, který byl v rámci této práce rozšířen o detailní model absorpce uhlovodíkových skupin. Pomocí dat z ERDA byly kalibračním multisample fitem čtyř vrstev určeny relativní síly přechodu pro valenční vibrace CH skupin, ty byly dále použity jednak pro analýzu celkové vodíkové koncentrace zbývajících vrstev a také pro analýzu rozložení vodíku mezi jednotlivé sp$x$CH$_y$ skupiny. 

Výsledky celkové vodíkové koncentrace získané pomocí optických měření se dobře shodují s výsledky získanými z ERDA. A i rozložení vodíku mezi jednotlivé skupiny odpovídá očekáváním. Ukázalo se, že většina vodíku je navázána na sp3 uhlík, což odpovídá teoretickým očekáváním, že vodík je potřeba ke stabilizaci sp3 fáze.

Dále jsou v výsledků jasně patrné změny v rozložení vodíku mezi jednotlivé skupiny v závislosti na jeho celkové koncentraci. Při nizkých koncentracích je vodík převážně přítomen v sp3CH a sp2CH skupinách, sp3CH$_2$ skupiny jsou přítomny v menším množství a vodíku vázaného v sp3CH$_3$ skupinách je minimum. S rostoucí celkovou koncentraci dochází k nárůstu množství sp3CH$_2$ a sp3CH$_3$ skupin. Všechny tyto výsledky velmi dobře odpovídají teoretickým očekáváním, pro nejtvrdší vrstvy s málo vodíkem je vodík rozmístěn rovnoměrně, naopak s rostoucí vodíkovou koncentraci roste počet koncových skupin, které přispívají k menšímu propojení uhlíkové mřížky a snižování hustoty a tvrdosti vrstev.

Také se ukázalo, že vrstvy jsou dobře teplotně stabilní. Žíháním vrstev až na 320\,$^\circ$ nedošlo z žádným výrazným změnám, pouze k malým fluktuacím celkové vodíkové koncentrace a mírnému přeskupení vodíku v rámci jednotlivých skupin.

Kromě celkové vodíkové koncentrace byly úspěšně určeny i další parametry vrstev, jako index lomu, parametry pásové struktury, některé parametry přechodových vrstev a substrátu.

Optické charakterizaci tenkých vrstev bych se chtěl i nadále věnovat při svém doktorském studiu, především se zaměřením na studium struktury pevných látek a jeho vlivu na optické charakteristiky. Z dalších věcí, které se do rozsahu této práce nevešly bych zmínil například rozbor optických defektů, jako nehomogenita, která právě u DLC vrstev hraje velkou roli a její zahrnutí do modelu by jistě vedlo k výrazně lepším výsledkům.

\cleardoublepage
