\documentclass{beamer}

\usepackage[czech]{babel}
\usepackage[utf8]{inputenc}
%\usepackage[plainpages=false,pdfpagelabels,unicode]{hyperref}
\usepackage{graphicx}

\usetheme{Warsaw}

\begin{document}

\title[Optical properties of dioxides by first principle calculations] % (optional, only for long titles)
{Optical properties of various dioxides by first principle calculations}
\subtitle{Evaluating the possibilities of Density Functional Theory}
\author{Pavel Ondračka}
\institute
{
	 Faculty of Science, Masaryk University\\
	Brno, Czech Republic
  \and
	CEITEC - Central European Institute of Technology\\
	Brno, Czech Republic
}
\date{17.6.2014}

\maketitle

\begin{frame}
	\frametitle{Outline}
    \tableofcontents
\end{frame}

\section{Motivation}
\begin{frame}
    \frametitle{Motivation}
    \framesubtitle{We want to calculate everything!}
	In theory we could use DFT calculations to:
	\begin{itemize}	
	\item calculate optical properties in broad spectral range (including core electron excitations)
	\item predict optical properties of new materials
	\item better understand underlying processes
	\item help development of new dispersion models
	\end{itemize}
\end{frame}

\section{Introduction}
\subsection{DFT}
\begin{frame}
    \frametitle{DFT introduction}
    \framesubtitle{It shoudl be easy? Right?}

	
	\begin{itemize}
	\item Ground state properties of a many-electron system are uniquely determined by an electron density
	\item Approximations - exchange-correlation functionals (LDA, GGA)
	\item WIEN2k - existing implementation of DFT, all electron full-potential scheme
	\end{itemize}
\end{frame}

\subsection{Optics}
\begin{frame}
    \frametitle{Selected optical definitions}
    \framesubtitle{What all those letters stand for?}

	There are few terms to describe optical response

	\begin{itemize}
	\item Refractive index $n$ and extinction coefficient $\kappa$, sometimes called complex refractive index $\tilde{n} = n + \mathrm{i} \kappa$
	\item Complex dielectric function $\tilde{\epsilon} = \epsilon_\mathrm{r} + \epsilon_\mathrm{i}$, or for nonisotropic materials dielectric tensor $\hat{\epsilon}$.
	\item $\epsilon_\mathrm{r} = n^2 - \kappa^2 $, $\epsilon_\mathrm{i} = 2 n \kappa$
	\item Absorption (attenuation) coefficient $\alpha = \frac{4 \pi \kappa}{\lambda}$ as known from Beer-Lambert law $I = I_0 \mathrm{e}^{-\alpha x}$  
	\end{itemize} 
	\small
	\begin{equation}	
	\epsilon_\mathrm{i} (E) = 
\left(\frac{eh}{m_\mathrm{e}E} \right)^2 \frac{1}{4 \pi \epsilon_0 \mathrm{B}_0} \sum_{j,k} | p_{j \rightarrow k} |^2
\int_{-\infty}^\infty f_\mathrm{e}(S) \mathcal{N}_j(S) f_\mathrm{h}(S+E) \mathcal{N}_k(S + E)\mathrm{d}S \text{,}
	\end{equation}
	\normalsize
\end{frame}

\begin{frame}
    \frametitle{Selected optical definitions}
    \framesubtitle{No more equations, I promise}

Kramers-Kronig relations:
\begin{equation}
\epsilon_\mathrm{r}(\omega) = 1 + \frac{1}{\pi} \mathcal{P} \int \frac{\epsilon_\mathrm{i}(\xi)}{\xi - \omega} \mathrm{d}\xi \mathrm{,}
\label{KKint}
\end{equation}

So called optical sum rule:
\begin{equation}
\int_0^\infty \epsilon_\mathrm{i} (\omega) \omega \mathrm{d} \omega = \frac{\pi}{2} \omega_\mathrm{p}^2 = \frac{\pi}{2} \frac{e^2 n_\mathrm{e}}{ \epsilon_0 m_\mathrm{e}} \mathrm{,}
\end{equation}

\end{frame}

\section{Selected problems}
\subsection{Correct band gap}

\begin{frame}
    \frametitle{Correct band gap}
    \framesubtitle{Its all about the potential}

	DFT is known to underestimate the band gap.

	Possible solutions:
	\begin{itemize}
	\item scissor operator
	\item orbital potential (LDA+U, GGA+U)
	\item mBJ exchange-correlation potential
	\end{itemize}
\end{frame}

\begin{frame}
    \frametitle{Correct band gap}
    \framesubtitle{mBJ seems to be the best solution}

    \begin{figure}
	\includegraphics[width=0.8\linewidth]{figures/compare.pdf}
	\end{figure}

\end{frame}

\begin{frame}
    \frametitle{Correct band gap}
    \framesubtitle{Correct band gap comes at cost}
	TiO$_2$ rutile band structure
	\begin{figure}
	\hspace{0.2cm} GGA \hspace{2.4cm} mBJ \hspace{2.8cm} G$_0$W$_0$\footnote{M Landmann, E Rauls, and W G Schmidt. Journal of Physics: Condensed Matter, 24(19):195503, 2012.}
	\includegraphics[height=3.8cm]{figures/TiO2-rutile-GGA.pdf}
	\includegraphics[height=3.8cm]{figures/TiO2-rutile-mBJ.pdf}
	\includegraphics[height=3.8cm]{figures/TiO2-rutile-GW.png}
	\end{figure}

\end{frame}

\subsection{Dielectric function in wide spectral range}
\begin{frame}
    \frametitle{Dielectric function in wide spectral range}
    \framesubtitle{Why is is important?}

	\begin{itemize}
	\item Advanced dispersion models can provide information such as mass density
	\item But we need information from outside spectral range of usual optical measurements 
	\item Also estimation of effective number of electrons is needed 
	\item DFT results can be used to enhance current (mostly empirical) models
	\end{itemize}	

\end{frame}

\begin{frame}
    \frametitle{Dielectric function in wide spectral range}
    \framesubtitle{HfO$_2$ example}

    \begin{figure}
	\includegraphics[width=\linewidth]{figures/HfO2-eps-overview.pdf}
	\end{figure}

\end{frame}

\subsection{Understanding the nature of electronic transitions}
\begin{frame}
    \frametitle{Understanding nature of electronic transitions}
    \framesubtitle{Another was to enhance dispersion models}

	\begin{columns}[c]
    \column{.4\textwidth}
	As was shown in (1), dielectric function is constructed by summing over all possible electron transitions, hence we can use this information to analyze them and reveal different transition types

    \column{.6\textwidth}
    \begin{figure}
	\includegraphics[width=\linewidth]{figures/CH30deconv.pdf}
	\end{figure}
	\end{columns}

\end{frame}

\begin{frame}
    \frametitle{Understanding nature of electronic transitions}
    \framesubtitle{Divide and conquer}

    \begin{figure}
	\includegraphics[height=6.3cm]{figures/cubic-eps.pdf}
	\end{figure}

\end{frame}

\subsection{Predictions of new materials}
\begin{frame}
    \frametitle{Predictions of new materials - Si$_x$Ti$_{1-x}$O$_2$}
    \framesubtitle{As long as its periodic, we can calculate it}

	\begin{columns}[c]
    \column{.5\textwidth}
     TiO$_2$-SiO$_2$ materials are very promising for optical applications, we used DFT to model Si$_x$Ti$_{1-x}$O$_2$ solid solutions for various $x$ values

    \begin{figure}
	\includegraphics[width=\linewidth]{figures/gap.pdf}
	\end{figure}
    \column{.5\textwidth}

    \begin{figure}
	\includegraphics[width=\linewidth]{figures/SiTiO2-eps.pdf}
	\end{figure}

    \end{columns}
\end{frame}


\begin{frame}
    \frametitle{Predictions of new materials - Si$_x$Ti$_{1-x}$O$_2$}
    \framesubtitle{Some interesting behavior showed up}

	\begin{columns}[c]
    \column{.4\textwidth}

    \begin{figure}
	\includegraphics[width=\linewidth]{figures/near-gap-eps.pdf}
	\newline
	\includegraphics[width=0.9\linewidth]{figures/cell.jpg}
	\end{figure}

    \column{.6\textwidth}

    \begin{figure}
	\includegraphics[height=4cm]{figures/spaghettiOSi.pdf}
	\includegraphics[height=4cm]{figures/spaghettinotOSi.pdf}
	\end{figure}

    \end{columns}

\end{frame}


\section{Conclusion}
\begin{frame}
    \frametitle{Conclusion}
    \framesubtitle{Just one more minute, and its finally over}
	\begin{center}Density functional theory\end{center}
	\begin{columns}[c]
    \column{.5\textwidth}
	\begin{itemize}
	\item Can be used to calculate band gap and optical constants 

	\item Can predict optical constants of new materials

	\item Can shed some light into nature of electronic transitions

	\item Can help in development of new dispersion models
	\end{itemize}

    \column{.5\textwidth}

	\begin{itemize}
	\item Is not able to calculate the sum rule properly

	\item Results are worse at high energies

	\item Can calculate only direct transitions

	\end{itemize}
	\end{columns}
\end{frame}

\begin{frame}
    \frametitle{Questions}
    \framesubtitle{Everything was clear, wasn't it?}

	\begin{center}
	Thank you for your attention.
	\end{center}
	\begin{center}
	Time for questions!
	\end{center}

\end{frame}

\end{document}
